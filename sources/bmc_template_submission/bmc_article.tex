 %% BioMed_Central_Tex_Template_v1.06
%%                                      %
%  bmc_article.tex            ver: 1.06 %
%                                       %

%%IMPORTANT: do not delete the first line of this template
%%It must be present to enable the BMC Submission system to
%%recognise this template!!

%%%%%%%%%%%%%%%%%%%%%%%%%%%%%%%%%%%%%%%%%
%%                                     %%
%%  LaTeX template for BioMed Central  %%
%%     journal article submissions     %%
%%                                     %%
%%          <8 June 2012>              %%
%%                                     %%
%%                                     %%
%%%%%%%%%%%%%%%%%%%%%%%%%%%%%%%%%%%%%%%%%

%%%%%%%%%%%%%%%%%%%%%%%%%%%%%%%%%%%%%%%%%%%%%%%%%%%%%%%%%%%%%%%%%%%%%
%%                                                                 %%
%% For instructions on how to fill out this Tex template           %%
%% document please refer to Readme.html and the instructions for   %%
%% authors page on the biomed central website                      %%
%% https://www.biomedcentral.com/getpublished                      %%
%%                                                                 %%
%% Please do not use \input{...} to include other tex files.       %%
%% Submit your LaTeX manuscript as one .tex document.              %%
%%                                                                 %%
%% All additional figures and files should be attached             %%
%% separately and not embedded in the \TeX\ document itself.       %%
%%                                                                 %%
%% BioMed Central currently use the MikTex distribution of         %%
%% TeX for Windows) of TeX and LaTeX.  This is available from      %%
%% https://miktex.org/                                             %%
%%                                                                 %%
%%%%%%%%%%%%%%%%%%%%%%%%%%%%%%%%%%%%%%%%%%%%%%%%%%%%%%%%%%%%%%%%%%%%%

%%% additional documentclass options:
%  [doublespacing]
%  [linenumbers]   - put the line numbers on margins

%%% loading packages, author definitions

%\documentclass[twocolumn]{bmcart}% uncomment this for twocolumn layout and comment line below
\documentclass{bmcart}

%%% Load packages
\usepackage{amsthm,amsmath}
%\RequirePackage[numbers]{natbib}
%\RequirePackage[authoryear]{natbib}% uncomment this for author-year bibliography
%\RequirePackage{hyperref}
\usepackage[utf8]{inputenc} %unicode support
%\usepackage[applemac]{inputenc} %applemac support if unicode package fails
%\usepackage[latin1]{inputenc} %UNIX support if unicode package fails

\usepackage{hyperref}

%%%%%%%%%%%%%%%%%%%%%%%%%%%%%%%%%%%%%%%%%%%%%%%%%
%%                                             %%
%%  If you wish to display your graphics for   %%
%%  your own use using includegraphic or       %%
%%  includegraphics, then comment out the      %%
%%  following two lines of code.               %%
%%  NB: These line *must* be included when     %%
%%  submitting to BMC.                         %%
%%  All figure files must be submitted as      %%
%%  separate graphics through the BMC          %%
%%  submission process, not included in the    %%
%%  submitted article.                         %%
%%                                             %%
%%%%%%%%%%%%%%%%%%%%%%%%%%%%%%%%%%%%%%%%%%%%%%%%%

\def\includegraphic{}
\def\includegraphics{}

%%% Put your definitions there:
\startlocaldefs
\endlocaldefs

%%% Begin ...
\begin{document}

%%% Start of article front matter
\begin{frontmatter}

\begin{fmbox}
\dochead{Research}

%%%%%%%%%%%%%%%%%%%%%%%%%%%%%%%%%%%%%%%%%%%%%%
%%                                          %%
%% Enter the title of your article here     %%
%%                                          %%
%%%%%%%%%%%%%%%%%%%%%%%%%%%%%%%%%%%%%%%%%%%%%%

\title{Physcraper: A Python package for continually updated phylogenetic trees using the Open Tree of Life}

%%%%%%%%%%%%%%%%%%%%%%%%%%%%%%%%%%%%%%%%%%%%%%
%%                                          %%
%% Enter the authors here                   %%
%%                                          %%
%% Specify information, if available,       %%
%% in the form:                             %%
%%   <key>={<id1>,<id2>}                    %%
%%   <key>=                                 %%
%% Comment or delete the keys which are     %%
%% not used. Repeat \author command as much %%
%% as required.                             %%
%%                                          %%
%%%%%%%%%%%%%%%%%%%%%%%%%%%%%%%%%%%%%%%%%%%%%%

\author[
  addressref={aff1},                   % id's of addresses, e.g. {aff1,aff2}
  corref={aff1},                       % id of corresponding address, if any
% noteref={n1},                        % id's of article notes, if any
  email={sanchez.reyes.luna@gmail.com}   % email address
]{\inits{L.L.}\fnm{Luna L.} \snm{Sanchez Reyes}}
\author[
  addressref={aff1,aff2},
  email={kandziom@natur.cuni.cz}
]{\inits{M.}\fnm{Martha} \snm{Kandziora}}
\author[
  addressref={aff1},
  corref={aff1},
  email={ejmctavish@ucmerced.edu}
]{\inits{E.J.}\fnm{Emily Jane} \snm{McTavish}}
%%%%%%%%%%%%%%%%%%%%%%%%%%%%%%%%%%%%%%%%%%%%%%
%%                                          %%
%% Enter the authors' addresses here        %%
%%                                          %%
%% Repeat \address commands as much as      %%
%% required.                                %%
%%                                          %%
%%%%%%%%%%%%%%%%%%%%%%%%%%%%%%%%%%%%%%%%%%%%%%

\address[id=aff1]{%                           % unique id
  \orgdiv{School of Natural Sciences},             % department, if any
  \orgname{University of California},          % university, etc
  \city{Merced},                              % city
  \cny{USA}                                    % country
}
\address[id=aff2]{%
  \orgdiv{Department of Botany, Faculty of Science},
  \orgname{Charles University},
  %\street{},
  %\postcode{}
  \city{Prague },
  \cny{Czech Republic}
}

%%%%%%%%%%%%%%%%%%%%%%%%%%%%%%%%%%%%%%%%%%%%%%
%%                                          %%
%% Enter short notes here                   %%
%%                                          %%
%% Short notes will be after addresses      %%
%% on first page.                           %%
%%                                          %%
%%%%%%%%%%%%%%%%%%%%%%%%%%%%%%%%%%%%%%%%%%%%%%

%\begin{artnotes}
%%\note{Sample of title note}     % note to the article
%\note[id=n1]{Equal contributor} % note, connected to author
%\end{artnotes}

\end{fmbox}% comment this for two column layout

%%%%%%%%%%%%%%%%%%%%%%%%%%%%%%%%%%%%%%%%%%%%%%%
%%                                           %%
%% The Abstract begins here                  %%
%%                                           %%
%% Please refer to the Instructions for      %%
%% authors on https://www.biomedcentral.com/ %%
%% and include the section headings          %%
%% accordingly for your article type.        %%
%%                                           %%
%% I followed sections from here:            %%
%% https://bmcbioinformatics.biomedcentral.com/submission-guidelines/preparing-your-manuscript/software-article
%%%%%%%%%%%%%%%%%%%%%%%%%%%%%%%%%%%%%%%%%%%%%%%

\begin{abstractbox}

\begin{abstract}
\parttitle{Background}
Phylogenies are a key part of research in many areas of biology. Tools that automate
some parts of the process of phylogenetic reconstruction, mainly molecular character matrix assembly,
have been developed for the advantage of both specialists in the field of phylogenetics and non-specialists.
However, interpretation of results, comparison with previously available phylogenetic
hypotheses, and selection of one phylogeny for downstream analyses and discussion still impose difficulties
to one that is not a specialist either on phylogenetic methods or on a particular group of study.

\parttitle{Results}
Physcraper is a command-line Python program that automates the update of published
phylogenies by adding public DNA sequences to underlying alignments of previously
published phylogenies. It also provides a framework for straightforward comparison
of published phylogenies with their updated versions, by leveraging upon tools from
the Open Tree of Life project to link taxonomic information across databases.
The program can be used by the nonspecialist, as a tool to generate phylogenetic
hypotheses based on publicly available expert phylogenetic knowledge.
Phylogeneticists and taxonomic group specialists will find it useful as a tool
to facilitate molecular dataset gathering and comparison
of alternative phylogenetic hypotheses (topologies).

\parttitle{Conclusions}
The Physcraper workflow showcases the benefits of doing open science for
phylogenetics, encouraging researchers to strive for better sharing practices.
Physcraper can be used with any OS and is released under an open-source license.
Detailed instructions for installation and
usage are available at \href{https://physcraper.readthedocs.io}{https://physcraper.readthedocs.io}.


\end{abstract}

%%%%%%%%%%%%%%%%%%%%%%%%%%%%%%%%%%%%%%%%%%%%%%
%%                                          %%
%% The keywords begin here                  %%
%%                                          %%
%% Put each keyword in separate \kwd{}.     %%
%%                                          %%
%%%%%%%%%%%%%%%%%%%%%%%%%%%%%%%%%%%%%%%%%%%%%%

\begin{keyword}
\kwd{gene tree}
\kwd{interoperability}
\kwd{open science}
\kwd{reproducibility}
\kwd{public database}
\kwd{DNA alignment}
\end{keyword}

% MSC classifications codes, if any
%\begin{keyword}[class=AMS]
%\kwd[Primary ]{}
%\kwd{}
%\kwd[; secondary ]{}
%\end{keyword}

\end{abstractbox}
%
%\end{fmbox}% uncomment this for two column layout

\end{frontmatter}

%%%%%%%%%%%%%%%%%%%%%%%%%%%%%%%%%%%%%%%%%%%%%%%%
%%                                            %%
%% The Main Body begins here                  %%
%%                                            %%
%% Please refer to the instructions for       %%
%% authors on:                                %%
%% https://www.biomedcentral.com/getpublished %%
%% and include the section headings           %%
%% accordingly for your article type.         %%
%%                                            %%
%% See the Results and Discussion section     %%
%% for details on how to create sub-sections  %%
%%                                            %%
%% use \cite{...} to cite references          %%
%%  \cite{koon} and                           %%
%%  \cite{oreg,khar,zvai,xjon,schn,pond}      %%
%%                                            %%
%%%%%%%%%%%%%%%%%%%%%%%%%%%%%%%%%%%%%%%%%%%%%%%%

%%%%%%%%%%%%%%%%%%%%%%%%% start of article main body
% <put your article body there>

%%%%%%%%%%%%%%%%
%% Background %%
%%
\section*{Background}
Phylogenies capture the shared history of organisms and provide key evolutionary
context for our biological observations.
Public biological databases constitute an amazing resource for evolutionary
studies. Updating existing phylogenies with molecular data that has never been
incorporated into any phylogenetic estimate, geographical location, fossils, and
other data in a reproducible and continuous manner is possible by establishing a
data interoperability framework for biological databases. Here, we introduce
Physcraper, a tool that automates database connections to build upon homology
hypotheses that taxon specialists have assessed and deemed appropriate for a
specific phylogenetic scope to update a starting tree and single locus alignments
with public DNA data.

Taxonomic idiosyncrasies across databases represent a huge challenge for automatic
integration of data into phylogenies, which can be addressed with a unified
taxonomy for name standardization. The Open Tree of Life project (OpenTree)
constructs a comprehensive tree of life by synthesizing published phylogenies
and taxonomy. OpenTree's "synthetic" tree comprises 2.3 million tips, of which
around 90,000 are supported by phylogenies - the remaining 1.4 million taxa are
placed in the tree based on taxonomy. To achieve this, OpenTree unifies taxonomic
data from various databases \cite{rees2017automated}, including the USA National
Center for Biodiversity Information (NCBI) molecular database GenBank
\cite{benson2000genbank}, among others. The OpenTree taxonomy represents a key resource
for connecting data from any biological database that has been integrated to it.

Another challenge for incorporating public molecular data into existing phylogenies
is assembling high-quality homology hypotheses.
While genomics has, and will continue to, revolutionize phylogenetic inference,
the variety of alternative genomic sequencing approaches it uses produce largely
non-overlapping genomic datasets across taxa, creating challenges in wide scale
phylogenetic reconstruction.

Phylogenomics ameliorate this problem by focusing on targeted capture of informative
loci \cite{andermann2020guide}. Yet, decades of single locus sequencing have generated
massive amounts of homologous DNA datasets that can be used for phylogenetic
reconstruction at many scales.

More than a decade ago, GenBank release 159 (April 15, 2007) already hosted 72
million DNA sequences that were gauged to have the potential to resolve phylogenetic
relationships of `r round(236023/240708, digits=4)*100`% of the almost 241,000
distinct taxa in the NCBI taxonomy at the time \cite{sanderson2008phylota}.
Assembling a DNA alignment from such a massive database can be done "by hand",
but it is a largely time consuming and mostly non-reproducible approach.
Computational pipelines that mine DNA databases fast, efficiently, and reproducibly,
have been applied to infer phylogenetic relationships in a variety of organisms
(e.g., \cite{smith2009mega, antonelli2017toward, izquierdo2014pumper}).
However, fine-grained curated markers and alignments can improve phylogenetic
reconstructions, even in phylogenomic analyses \cite{fragoso2017pilot}.

There are almost 8,200 publicly available, peer-reviewed alignments, covering
around 100,000 distinct taxa in the TreeBASE database \cite{piel2009treebase}, which
can be used as seeds to mine molecular databases, and as "jump-start" alignments
for phylogenetic reconstructions \cite{morrison2006multiple} to continually enrich,
update and compare existing phylogenetic knowledge.

Physcraper is a Python pipeline using OpenTree's taxonomy and programmatic access
protocols (API's) to implement a database interoperability framework that automatically
links phylogenies that have been standardized to OpenTree taxonomy, to alignments
from TreeBASE, data from GenBank, and phylogenies from OpenTree's Phylesystem.
Physcraper aims to demonstrate the benefits of reproducible workflows and open
science in phylogenetics, and encourage better data sharing practices in the community.


\section*{Implementation}

The general Physcraper framework (Figure 1) consists of 4 steps: 1)identifying and
processing a tree and its underlying alignment; 2) performing a BLAST search of
DNA sequences from original alignment on GenBank, and filtering of new sequences;
3) profile-aligning new sequences to original alignment; 4) performing a phylogenetic
analysis and comparing the updated tree to existing phylogenies.

\subsection*{The inputs: a tree and an alignment}

Taxon names in the input tree must be standardized to OpenTree taxonomy
\cite{rees2017automated} using OpenTree's bulk Taxonomic Name Resolution Service
\href{https://tree.opentreeoflife.org/curator/tnrs/}{TNRS} tool. Users can upload their
own tree, or choose from among the 2, 950 standardized trees stored in OpenTree's
\href{https://github.com/opentreeoflife/phylesystem}{Phylesystem} that also have
alignments available on TreeBASE \cite{piel2009treebase}.

The input alignment is a single locus DNA dataset that was used in part or in
whole to generate the input tree. Physcraper retrieves TreeBASE alignments
automatically. Alternatively, users must provide the path to a local copy of the
alignment.
Only taxa that are both in the sequence alignment and in the tree are considered
further for analysis; at least one taxon and its corresponding sequence are required.

\subsection*{DNA sequence search and filtering}

The Basic Local Alignment Search Tool, BLAST \cite{altschul1990basic} is used for DNA
sequence search on a remote or local GenBank database. It is constrained to a
"search taxon", a taxonomic group in the NCBI taxonomy that is automatically
identified using the OpenTree
\href{<https://github.com/OpenTreeOfLife/germinator/wiki/Taxonomy-API-v3#mrca>}{API}
\cite{rees2017automated}, as the most recent common ancestor of ingroup taxa that is
also a named clade in the NCBI taxonomy (Figure 1).

BLAST is implemented with the `blastn` function \cite{camacho2009blast} and the BioPython
\cite{cock2009biopython} BLAST function from
\href{https://biopython.org/DIST/docs/api/Bio.Blast.NCBIWWW-module.html}{NCBIWWW module}
modified to accept an alternative BLAST address.
Each sequence in the alignment is BLASTed once against all DNA sequences in GenBank.
New sequences are excluded for analysis if they 1) are not in the search taxon;
2) have an e-value above the cutoff (default to 0.00001); 3) fall outside a min
and max length threshold, defined as the proportion of the average length without
gaps of all sequences in input alignment (default values of 80% and 120%,
respectively); 4) or if they are either identical to or shorter than an existing
sequence in the input alignment and they represent the same taxon in OpenTree or
NCBI taxonomy.
An arbitrary maximum number of randomly chosen sequences per taxon are allowed
(default to 5).

Reverse, complement, and reverse-complement sequences are identified and translated
using BioPython internal functions \cite{cock2009biopython}.
Iterative cycles of BLAST searches can be performed, by blasting all new sequences
until no new ones are found. By default only one BLAST cycle is performed.

\subsection*{New DNA sequence alignment}

MUSCLE \cite{edgar2004muscle} is used to perform a profile alignment in which the
original alignment is used as a template of homology criteria to align new sequences.
The final alignment is not further automatically checked, and additional inspection
and refinement are recommended.

\subsection*{Tree reconstruction and comparison}

RAxML \cite{stamatakis2014raxml} is implemented to reconstruct a Maximum Likelihood
(ML) gene tree for each input alignment with default settings (GTRCAT model and
100 bootstrap replicates with default algorithm), using input tree as starting
tree for ML searches.
Bootstrap results are summarized using DendroPy's SumTrees module
\cite{sukumaran2010dendropy}.

Physcraper's main result is an updated phylogenetic hypothesis for the search taxon.
Updated and original tree are compared with Robinson-Foulds weighted and unweighted
metrics estimated with Dendropy \cite{sukumaran2010dendropy}, and with a node by node
comparison between the synthetic OpenTree and original and updated tree individually,
using OpenTree's conflict API \cite{redelings2017supertree}.



\section*{Results}
\subsection*{Case Study: The hollies}

A user is interested in phylogenetic relationships within the genus *Ilex*. Commonly
known as "hollies", the genus encompasses between 400-700 living species, and is
the only extant clade within the family Aquifoliaceae, order Aquifoliales of flowering
plants.

An online literature review in June 2020 (Google scholar search for "ilex phylogeny")
reveals that there are several published phylogenies showing relationships within
the hollies \cite{cuenoud2000molecular, manen2010history, setoguchi2000intersectional,
selbach2009new}, but only two have data publicly available \cite{gottlieb2005molecular, yao2020phylogeny}.
\cite{gottlieb2005molecular} made original tree and alignment available in
\href{https://treebase.org/treebase-web/search/study/summary.html?id=1091}{TreeBASE study 1091}.
The tree sampling 41 species was also added to
OpenTree's Phylesystem \href{https://tree.opentreeoflife.org/curator/study/edit/pg_2827/?tab=home}{study pg\_2827}
and it has been integrated into OpenTree's synthetic tree
\href{https://devtree.opentreeoflife.org/opentree/opentree12.3@mrcaott68451ott89474/Ilex-theizans--Ilex-dumosa}{here}.

The most recent *Ilex* tree \cite{yao2020phylogeny} is available in
OpenTree's Phylesystem \href{https://tree.opentreeoflife.org/curator/study/view/ot_1984}{study ot\_1984}
and in the DRYAD repository \cite{yao2020dryad}.
With 175 tips, the \cite{yao2020phylogeny} tree is the best sampled phylogeny yet available for
the hollies.

We ran Physcraper on a laptop Linux computer to update an internal transcribed
spacer DNA region (ITS) alignment from \cite{gottlieb2005molecular}, using a local
GenBank database.  BLAST and RAxML analyses ran for 19hrs 45min, with bootstrap
analyses taking an additional 13hrs.
<!-- **MTH: you probably need some details about the hardware, given the fact
hat you are discussing running times** -->
The updated \cite{gottlieb2005molecular} tree (Fig. \@ref(fig:results)) displays
all 41 distinct
taxa from the original study plus 231 new tips, contributing phylogenetic data to
84 additional *Ilex* taxa. The best RaxML tree is 99% resolved, with 25% of nodes
with bootstrap support < 0.1 and 48% nodes with bootstrap support > 0.75.
A large portion of internal branches are negligibly small, with 30 branches
< 0.00001 substitution rate units, from which only 9 have a bootstrap support > 0.75
(Figure 2).
For comparison, \cite{yao2020phylogeny} also contains all 41 distinct taxa from
the original
\cite{gottlieb2005molecular} study,
and contributes phylogenetic data to 134 additional *Ilex* taxa, from which
`r 135-68` are also in updated \cite{gottlieb2005molecular}. While \cite{yao2020phylogeny} also used
ITS as a marker, their GenBank data is not released yet, so Physcraper was unable
to incorporate 68 additional taxa into the analysis. However, Physcraper was able
to incorporate 18 taxa that were not in \cite{yao2020phylogeny}.

\section*{Discussion}

Databases preserving and democratizing access to biological data
have become essential resources for science.
New molecular data keep accumulating and tools facilitating its integration into
existent evolutionary knowledge are needed.

Phylogenetic pipelines designed to make evolutionary sense of the vast amount of
public molecular data (e.g., Phylota \cite{sanderson2008phylota},
PHLAWD \cite{smith2009mega},
SUPERSMART \cite{antonelli2017toward}) focus on generating full phylogenies *de novo*,
i.e., inferring phylogenetic relationships from a newly generated homology hypothesis,
as opposed to e.g., supertrees, that are generated by summarizing previous
phylogenetic estimates.
While Physcraper does not generate phylogenies *de novo* in a traditional sense,
it successfully generates new phylogenetic knowledge, revealing the importance of
open science in facilitating phylogenetic placement of public molecular data and
accelerating enrichment and updating of phylogenetic relationships in any region
of the tree of life.
The PUMPER pipeline \cite{izquierdo2014pumper} also uses the concept of updating
pre-existing alignments to incorporate public molecular data into phylogenies.
Unfortunately, installation was unsuccessful following instructions from the author,
and a comparison analysis between Physcraper and PUMPER is unfeasible at present.
%% **MTH: add some more details about what aspect of the PUMPER install failed**

Physcraper generates individual gene trees, failing to capture the complexity of
species' evolutionary history \cite{song2012resolving}. Yet, Physcraper facilitates
gathering alignments and gene trees for multiple loci from a group of interest,
that can be used to reconstruct species trees with ASTRAL \cite{mirarab2014astral},
BEAST2 \cite{bouckaert2019beast}, or SVD Quartets \cite{chifman2014quartet}).

Physcraper can potentially link phylogenies to data available in any of the
taxonomies integrated in the OpenTree taxonomy \cite{rees2017automated}, such as
geographical locations from the Global Biodiversity Information Facility, or
fossils from the Paleobiology Database.
%% a resource that is contributing to achieve the goal of "exploring and
%% analysing biodiversity at an accelerated pace, and returning systematics into
%% the mainstream of science" \cite{wilson2003encyclopedia}.
The Physcraper workflow can be used to rapidly (in a matter of hours)
address challenges overarching both fields of ecology and evolution, such as
phylogenetically placing newly discovered species \cite{webb2010biodiversity},
systematizing molecular (and other) databases, i.e., curating taxonomic assignations
\cite{san2010molecular},
and generating custom trees for ecological \cite{helmus2012phylogenetic} and
evolutionary downstream analyses \cite{stoltzfus2013phylotastic}.

Data repositories hold more information than meets the eye.
Beyond the main data, they are rich sources of metadata that can be leveraged
for the advantage of all areas of biology as well as the advancement of scientific
policy and applications.
Initial ideas about the data are constantly changed by results from new analyses.
Physcraper provides a framework for reproducible phylogenetics that has the
potential to consistently provide context for these ideas, highlighting the
importance of data sharing and open science in the field, biology and science.


\section*{Conclusions}
Text for this section\ldots

\section*{Availability and requirements}

\textbf{Project name:} Physcraper

\textbf{Project home page:} \href{https://physcraper.readthedocs.io/en/latest/index.html}{https://physcraper.readthedocs.io/en/latest/index.html}

\textbf{Operating System:} Linux, Mac, Windows

\textbf{Programming Language:} Python

\textbf{Other requirements:} Dependencies

\textbf{License:} GNU

\textbf{Any restrictions to use by non-academics:} As specified by the License

\section*{Abbreviations}

\textbf{OpenTree:} The Open Tree of Life project

%%%%%%%%%%%%%%%%%%%%%%%%%%%%%%%%%%%%%%%%%%%%%%
%%                                          %%
%% Backmatter begins here                   %%
%%                                          %%
%%%%%%%%%%%%%%%%%%%%%%%%%%%%%%%%%%%%%%%%%%%%%%
\section*{Declarations}

\begin{backmatter}

\section*{Ethics approval and consent to participate}
Not applicable.

\section*{Consent for publication}
Not applicable.

\section*{Availability of data and materials}

The datasets generated and analysed during the current study are available in
the repositories "physcraper" containing the source code,
\href{https://github.com/McTavishLab/physcraper}{https://github.com/McTavishLab/physcraper};
"physcraperex" containing the examples,
\href{https://github.com/McTavishLab/physcraperex}{https://github.com/McTavishLab/physcraperex};
and, "physcraper\_ms" containing this reproducible manuscript,
\href{https://github.com/McTavishLab/physcraper\_ms}{https://github.com/McTavishLab/physcraper\_ms}.

\section*{Competing interests}
The authors declare that they have no competing interests.

\section*{Funding}
This research was supported by the grant "Sustaining the Open Tree of Life", NSF ABI
No. 1759838, and ABI No. 1759846.
Computer time was provided by the Multi-Environment Research Computer for
Exploration and Discovery (MERCED) cluster from the University of California,
Merced (UCM), supported by the NSF Grant No. ACI-1429783.

\section*{Authors' contributions}
LLSR wrote manuscript, alignment code, documentation, performed analyses and developed examples; MK wrote code for ncbidataparser module, filtering of sequences per OTU and using offline blast searches, wrote documentation and tests; EJM conceived study, wrote most of the code, documentation and tests.
All authors contributed to the manuscript and gave final approval for publication.
\ldots

\section*{Acknowledgements}
We thank the members of the OpenTree development team and the "short bar" Science
and Engineering Building 1, UCM, joint lab paper discussion group for valuable
comments on this manuscript.


%%%%%%%%%%%%%%%%%%%%%%%%%%%%%%%%%%%%%%%%%%%%%%%%%%%%%%%%%%%%%
%%                  The Bibliography                       %%
%%                                                         %%
%%  Bmc_mathpys.bst  will be used to                       %%
%%  create a .BBL file for submission.                     %%
%%  After submission of the .TEX file,                     %%
%%  you will be prompted to submit your .BBL file.         %%
%%                                                         %%
%%                                                         %%
%%  Note that the displayed Bibliography will not          %%
%%  necessarily be rendered by Latex exactly as specified  %%
%%  in the online Instructions for Authors.                %%
%%                                                         %%
%%%%%%%%%%%%%%%%%%%%%%%%%%%%%%%%%%%%%%%%%%%%%%%%%%%%%%%%%%%%%

% if your bibliography is in bibtex format, use those commands:
\bibliographystyle{bmc-mathphys} % Style BST file (bmc-mathphys, vancouver, spbasic).
\bibliography{bmc_article}      % Bibliography file (usually '*.bib' )
% for author-year bibliography (bmc-mathphys or spbasic)
% a) write to bib file (bmc-mathphys only)
% @settings{label, options="nameyear"}
% b) uncomment next line
%\nocite{label}

% or include bibliography directly:
% \begin{thebibliography}
% \bibitem{b1}
% \end{thebibliography}

%%%%%%%%%%%%%%%%%%%%%%%%%%%%%%%%%%%
%%                               %%
%% Figures                       %%
%%                               %%
%% NB: this is for captions and  %%
%% Titles. All graphics must be  %%
%% submitted separately and NOT  %%
%% included in the Tex document  %%
%%                               %%
%%%%%%%%%%%%%%%%%%%%%%%%%%%%%%%%%%%

%%
%% Do not use \listoffigures as most will included as separate files

\section*{Figures}
  \begin{figure}[h!]
  \caption{The Physcraper framework consists of 4 general steps (see text). The software is fully
  described on its documentation website at \href{https://physcraper.readthedocs.io}{https://physcraper.readthedocs.io},
  along with installation instructions, function usage descriptions, examples and tutorials.
}
\end{figure}

\begin{figure}[h!]
  \caption{A) Phylogeny updated with Physcraper from original
  \cite{gottlieb2005molecular} tree in B.
  Tips in original alignment and new tips added with Physcraper are depicted in
  black and red, respectively. Physcraper obtained sequences from the GenBank
  database via local BLAST of all sequences in the original alignment that generated
  tree in B), filtered them following criteria from section "DNA sequence search and
  filtering", aligned them to original alignment using MUSCLE and performed a
  phylogenetic reconstruction using RAxML with 100 bootstraps. B-D conflict analyses
  performed with OpenTree tools.}
\end{figure}

%%%%%%%%%%%%%%%%%%%%%%%%%%%%%%%%%%%
%%                               %%
%% Tables                        %%
%%                               %%
%%%%%%%%%%%%%%%%%%%%%%%%%%%%%%%%%%%

%% Use of \listoftables is discouraged.
%%
\section*{Tables}
\begin{table}[h!]
\caption{Sample table title. This is where the description of the table should go}
  \begin{tabular}{cccc}
    \hline
    & B1  &B2   & B3\\ \hline
    A1 & 0.1 & 0.2 & 0.3\\
    A2 & ... & ..  & .\\
    A3 & ..  & .   & .\\ \hline
  \end{tabular}
\end{table}

%%%%%%%%%%%%%%%%%%%%%%%%%%%%%%%%%%%
%%                               %%
%% Additional Files              %%
%%                               %%
%%%%%%%%%%%%%%%%%%%%%%%%%%%%%%%%%%%

\section*{Additional Files}
  \subsection*{Additional file 1 --- Sample additional file title}
    Additional file descriptions text (including details of how to
    view the file, if it is in a non-standard format or the file extension).  This might
    refer to a multi-page table or a figure.

  \subsection*{Additional file 2 --- Sample additional file title}
    Additional file descriptions text.

\end{backmatter}
\end{document}
