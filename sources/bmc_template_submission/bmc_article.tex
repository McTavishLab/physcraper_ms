%% BioMed_Central_Tex_Template_v1.06
%%                                      %
%  bmc_article.tex            ver: 1.06 %
%                                       %

%%IMPORTANT: do not delete the first line of this template
%%It must be present to enable the BMC Submission system to
%%recognise this template!!

%%%%%%%%%%%%%%%%%%%%%%%%%%%%%%%%%%%%%%%%%
%%                                     %%
%%  LaTeX template for BioMed Central  %%
%%     journal article submissions     %%
%%                                     %%
%%          <8 June 2012>              %%
%%                                     %%
%%                                     %%
%%%%%%%%%%%%%%%%%%%%%%%%%%%%%%%%%%%%%%%%%

%%%%%%%%%%%%%%%%%%%%%%%%%%%%%%%%%%%%%%%%%%%%%%%%%%%%%%%%%%%%%%%%%%%%%
%%                                                                 %%
%% For instructions on how to fill out this Tex template           %%
%% document please refer to Readme.html and the instructions for   %%
%% authors page on the biomed central website                      %%
%% https://www.biomedcentral.com/getpublished                      %%
%%                                                                 %%
%% Please do not use \input{...} to include other tex files.       %%
%% Submit your LaTeX manuscript as one .tex document.              %%
%%                                                                 %%
%% All additional figures and files should be attached             %%
%% separately and not embedded in the \TeX\ document itself.       %%
%%                                                                 %%
%% BioMed Central currently use the MikTex distribution of         %%
%% TeX for Windows) of TeX and LaTeX.  This is available from      %%
%% https://miktex.org/                                             %%
%%                                                                 %%
%%%%%%%%%%%%%%%%%%%%%%%%%%%%%%%%%%%%%%%%%%%%%%%%%%%%%%%%%%%%%%%%%%%%%

%%% additional documentclass options:
%  [doublespacing]
%  [linenumbers]   - put the line numbers on margins

%%% loading packages, author definitions

%\documentclass[twocolumn]{bmcart}% uncomment this for twocolumn layout and comment line below
\documentclass{bmcart}

%%% Load packages
\usepackage{amsthm,amsmath}
%\RequirePackage[numbers]{natbib}
%\RequirePackage[authoryear]{natbib}% uncomment this for author-year bibliography
\RequirePackage{hyperref}
\usepackage[utf8]{inputenc} %unicode support
%\usepackage[applemac]{inputenc} %applemac support if unicode package fails
%\usepackage[latin1]{inputenc} %UNIX support if unicode package fails

%\usepackage{hyperref}

%%%%%%%%%%%%%%%%%%%%%%%%%%%%%%%%%%%%%%%%%%%%%%%%%
%%                                             %%
%%  If you wish to display your graphics for   %%
%%  your own use using includegraphic or       %%
%%  includegraphics, then comment out the      %%
%%  following two lines of code.               %%
%%  NB: These line *must* be included when     %%
%%  submitting to BMC.                         %%
%%  All figure files must be submitted as      %%
%%  separate graphics through the BMC          %%
%%  submission process, not included in the    %%
%%  submitted article.                         %%
%%                                             %%
%%%%%%%%%%%%%%%%%%%%%%%%%%%%%%%%%%%%%%%%%%%%%%%%%

\def\includegraphic{}
\def\includegraphics{}

%%% Put your definitions there:
\startlocaldefs
\endlocaldefs

%%% Begin ...
\begin{document}

%%% Start of article front matter
\begin{frontmatter}

\begin{fmbox}
\dochead{Research}

%%%%%%%%%%%%%%%%%%%%%%%%%%%%%%%%%%%%%%%%%%%%%%
%%                                          %%
%% Enter the title of your article here     %%
%%                                          %%
%%%%%%%%%%%%%%%%%%%%%%%%%%%%%%%%%%%%%%%%%%%%%%

\title{Physcraper: A Python package for continually updated phylogenetic trees using the Open Tree of Life}

%%%%%%%%%%%%%%%%%%%%%%%%%%%%%%%%%%%%%%%%%%%%%%
%%                                          %%
%% Enter the authors here                   %%
%%                                          %%
%% Specify information, if available,       %%
%% in the form:                             %%
%%   <key>={<id1>,<id2>}                    %%
%%   <key>=                                 %%
%% Comment or delete the keys which are     %%
%% not used. Repeat \author command as much %%
%% as required.                             %%
%%                                          %%
%%%%%%%%%%%%%%%%%%%%%%%%%%%%%%%%%%%%%%%%%%%%%%

\author[
  addressref={aff1},                   % id's of addresses, e.g. {aff1,aff2}
  corref={aff1},                       % id of corresponding address, if any
% noteref={n1},                        % id's of article notes, if any
  email={sanchez.reyes.luna@gmail.com}   % email address
]{\inits{L.L.}\fnm{Luna L.} \snm{Sanchez Reyes}}
\author[
  addressref={aff1,aff2},
  email={kandziom@natur.cuni.cz}
]{\inits{M.}\fnm{Martha} \snm{Kandziora}}
\author[
  addressref={aff1},
  corref={aff1},
  email={ejmctavish@ucmerced.edu}
]{\inits{E.J.}\fnm{Emily Jane} \snm{McTavish}}
%%%%%%%%%%%%%%%%%%%%%%%%%%%%%%%%%%%%%%%%%%%%%%
%%                                          %%
%% Enter the authors' addresses here        %%
%%                                          %%
%% Repeat \address commands as much as      %%
%% required.                                %%
%%                                          %%
%%%%%%%%%%%%%%%%%%%%%%%%%%%%%%%%%%%%%%%%%%%%%%

\address[id=aff1]{%                           % unique id
  \orgdiv{School of Natural Sciences},             % department, if any
  \orgname{University of California},          % university, etc
  \city{Merced},                              % city
  \cny{USA}                                    % country
}
\address[id=aff2]{%
  \orgdiv{Department of Botany, Faculty of Science},
  \orgname{Charles University},
  %\street{},
  %\postcode{}
  \city{Prague },
  \cny{Czech Republic}
}

%%%%%%%%%%%%%%%%%%%%%%%%%%%%%%%%%%%%%%%%%%%%%%
%%                                          %%
%% Enter short notes here                   %%
%%                                          %%
%% Short notes will be after addresses      %%
%% on first page.                           %%
%%                                          %%
%%%%%%%%%%%%%%%%%%%%%%%%%%%%%%%%%%%%%%%%%%%%%%

%\begin{artnotes}
%%\note{Sample of title note}     % note to the article
%\note[id=n1]{Equal contributor} % note, connected to author
%\end{artnotes}

\end{fmbox}% comment this for two column layout

%%%%%%%%%%%%%%%%%%%%%%%%%%%%%%%%%%%%%%%%%%%%%%%
%%                                           %%
%% The Abstract begins here                  %%
%%                                           %%
%% Please refer to the Instructions for      %%
%% authors on https://www.biomedcentral.com/ %%
%% and include the section headings          %%
%% accordingly for your article type.        %%
%%                                           %%
%% I followed sections from here:            %%
%% https://bmcbioinformatics.biomedcentral.com/submission-guidelines/preparing-your-manuscript/software-article
%%%%%%%%%%%%%%%%%%%%%%%%%%%%%%%%%%%%%%%%%%%%%%%

\begin{abstractbox}

\begin{abstract}
\parttitle{Background}
Phylogenies are a key part of research in many areas of biology. Tools that automate
some parts of the process of phylogenetic reconstruction, mainly molecular character matrix assembly,
have been developed for the advantage of both specialists in the field of phylogenetics and non-specialists.
However, interpretation of results, comparison with previously available phylogenetic
hypotheses, and selection of one phylogeny for downstream analyses and discussion still impose difficulties
to one that is not a specialist either on phylogenetic methods or on a particular group of study.

\parttitle{Results}
Physcraper is a command-line Python program that automates the update of published
phylogenies by adding public DNA sequences to underlying alignments of previously
published phylogenies. It also provides a framework for straightforward comparison
of published phylogenies with their updated versions, by leveraging upon tools from
the Open Tree of Life project to link taxonomic information across databases.
The program can be used by the nonspecialist, as a tool to generate phylogenetic
hypotheses based on publicly available expert phylogenetic knowledge.
Phylogeneticists and taxonomic group specialists will find it useful as a tool
to facilitate molecular dataset gathering and comparison
of alternative phylogenetic hypotheses (topologies).

\parttitle{Conclusions}
The Physcraper workflow showcases the benefits of doing open science for
phylogenetics, encouraging researchers to strive for better sharing practices.
Physcraper can be used with any OS and is released under an open-source license.
Detailed instructions for installation and
usage are available at \href{https://physcraper.readthedocs.io}{https://physcraper.readthedocs.io}.


\end{abstract}

%%%%%%%%%%%%%%%%%%%%%%%%%%%%%%%%%%%%%%%%%%%%%%
%%                                          %%
%% The keywords begin here                  %%
%%                                          %%
%% Put each keyword in separate \kwd{}.     %%
%%                                          %%
%%%%%%%%%%%%%%%%%%%%%%%%%%%%%%%%%%%%%%%%%%%%%%

\begin{keyword}
\kwd{gene tree}
\kwd{interoperability}
\kwd{open science}
\kwd{reproducibility}
\kwd{public database}
\kwd{DNA alignment}
\end{keyword}

% MSC classifications codes, if any
%\begin{keyword}[class=AMS]
%\kwd[Primary ]{}
%\kwd{}
%\kwd[; secondary ]{}
%\end{keyword}

\end{abstractbox}
%
%\end{fmbox}% uncomment this for two column layout

\end{frontmatter}

%%%%%%%%%%%%%%%%%%%%%%%%%%%%%%%%%%%%%%%%%%%%%%%%
%%                                            %%
%% The Main Body begins here                  %%
%%                                            %%
%% Please refer to the instructions for       %%
%% authors on:                                %%
%% https://www.biomedcentral.com/getpublished %%
%% and include the section headings           %%
%% accordingly for your article type.         %%
%%                                            %%
%% See the Results and Discussion section     %%
%% for details on how to create sub-sections  %%
%%                                            %%
%% use \cite{...} to cite references          %%
%%  \cite{koon} and                           %%
%%  \cite{oreg,khar,zvai,xjon,schn,pond}      %%
%%                                            %%
%%%%%%%%%%%%%%%%%%%%%%%%%%%%%%%%%%%%%%%%%%%%%%%%

%%%%%%%%%%%%%%%%%%%%%%%%% start of article main body
% <put your article body there>

%%%%%%%%%%%%%%%%
%% Background %%
%%
\section*{Background}
Phylogenies capture the shared history of organisms and provide key evolutionary
context for our biological observations \cite{dobzhansky1973nothing}.
Updating existing phylogenies with publicly available molecular sequence data that
has never been
incorporated into any phylogenetic estimate provides the opportunity to study
the evolutionary history of many taxa in a reproducible and continuous manner.
Here, we introduce Physcraper, a tool that establishes a data interoperability
framework for biological databases to automate data connections across databases,
with the main goal of building upon published alignments to extend existing
phylogenetic inferences with more data and taxa.
Physcraper updates a starting tree and single locus alignments
with public DNA data, and links the tips in these trees to a unified, interoperable
taxonomic resource \cite{rees2017automated}.

Data such as
geographical location, fossil ranges, and genetic and phenotypic information increasingly
available in public databases constitute an amazing
resource for biological discovery \cite{baxevanis2015importance}.
One of the main challenges for automatic
integration of biological data across databases are varying taxonomic idiosyncrasies.
To address this challenge, the Open Tree of Life project (OpenTree)
created a unified taxonomy for name standardization, by integrating taxonomic
data from several databases \cite{rees2017automated}, including the USA National
Center for Biodiversity Information (NCBI) taxonomy \cite{federhen_ncbi_2012, schoch_ncbi_2020},
and the Global Biodiversity Information Facility (GBIF) \cite{gbif_secretariat_gbif_2019} among others.
By using the existing OpenTree taxonomy programmatic tools to map tip names,
Physcraper has a framework for connecting updated phylogenies with data from any
biological database.

Decades of single locus sequencing have generated
massive amounts of homologous DNA datasets that have the potential to be used for phylogenetic
reconstruction at many scales \cite{sanderson2008phylota}.
More than a decade ago, GenBank release 159 (April 15, 2007) already hosted 72
million DNA sequences that were gauged to have the potential to resolve phylogenetic
relationships of 98.05\% of the almost 241,000
distinct taxa in the NCBI taxonomy at the time \cite{sanderson2008phylota}.
However, even thirteen years later, phylogenetic estimates for many of these taxa are not available \cite{mctavish_how_2017}.
OpenTree's comprehensive tree of life comprises 2.3 million tips, of which around
90,000 are supported by phylogenies - the remaining 1.4 million taxa are placed
in the tree based on taxonomy.
There is a considerable amount of phylogenetically
informative data in GenBank with the potential to fill these phylogenetic gaps in
the tree of life, but this data either has not been analysed or the analyses
have not been made publicly available \cite{mctavish_how_2017}.

Assembling a DNA alignment from such a massive database as GenBank can be done "by hand",
but that is a time consuming approach which is not highly reproducible.
A variety of computational pipelines that mine DNA databases fast, efficiently, and reproducibly
have been developed and used to infer phylogenetic relationships in a variety of organisms
(e.g., \cite{smith2009mega, antonelli2017toward, izquierdo2014pumper, pearse2013phylogenerator}).
While genomics has, and will continue to, revolutionize phylogenetic inference,
the diversity of alternative genomic sequencing approaches implemented produce
largely non-overlapping homology hypotheses across taxa \cite{jones2016targeted},
creating challenges for phylogenetic reconstruction.
Phylogenomics addresses this problem by focusing on targeted capture of informative
regions \cite{andermann2020guide}.
However, fine-grained curated markers and alignments can significantly improve phylogenetic
reconstructions, even in phylogenomic analyses \cite{fragoso2017pilot}.

Physcraper improves on previous work in automating phylogenetic reconstruction by leveraging
the power of existing homology hypotheses that taxon specialists have assessed
and deemed appropriate for a specific phylogenetic scope.
There are almost 8,200 publicly available, peer-reviewed curated alignments, covering
around 100,000 distinct taxa in the TreeBASE database \cite{piel2009treebase}, which
can be leveraged as seeds to mine molecular databases, and as ``jump-start'' alignments
for phylogenetic reconstructions \cite{morrison2006multiple} to continually enrich,
update and compare existing phylogenetic knowledge.

Physcraper is implemented as a Python pipeline that uses OpenTree's programmatic access
protocols (API's) to automatically
link any phylogeny mapped to OpenTree's standardized taxonomy \cite{mctavish2015phylesystem},
to alignments
from TreeBASE \cite{sanderson1994treebase}, and data from GenBank \cite{benson2000genbank}.
Its utility and functionalities are presented with a case-study analysis of a
group of flowering plants, the hollies.


\section*{Implementation}

Physcraper is implemented with Python and can also be run on the command line.
It currently consists of 12 modules.
%% , classes, methods and functions.%%
For testing and improving the quality of the Python code syntax in the Phsycraper code,
we used the Pylint \cite{pylintWeb, pylintManual} software with a configuration file.
We improved the code syntax on modules with low Pylint scores and fixed errors by
following Pylint's recommendations.
In some cases, we overruled Pylint's recommendations by using its check-disabling
system. As of now, all modules have a Pylint score of 10/10.

The general Physcraper framework (Figure 1) consists of 4 steps: 1) identifying and
processing a tree and its underlying alignment; 2) performing a BLAST search of
DNA sequences from original alignment on GenBank, and filtering of new sequences;
3) profile-aligning new sequences to original alignment; 4) performing a phylogenetic
analysis and comparing the updated tree to existing phylogenies.

\subsection*{The inputs: a tree and an alignment}

Taxon names in the input tree must be standardized to OpenTree taxonomy
\cite{rees2017automated} using OpenTree's bulk Taxonomic Name Resolution Service
(TNRS) tool \cite{TNRStool}. Users can upload their
own tree, or choose from among the 2, 950 standardized trees stored in OpenTree's
Phylesystem \cite{phylesystemGithub, mctavish2015phylesystem} that also have
alignments available on TreeBASE \cite{piel2009treebase}.

The input alignment is a single locus DNA dataset that was used in part or in
whole to generate the input tree. Physcraper retrieves TreeBASE alignments
automatically. Alternatively, users must provide the path to a local copy of the
alignment.
Only taxa that are both in the sequence alignment and in the tree are considered
further for analysis; at least one taxon and its corresponding sequence are required.

\subsection*{DNA sequence search and filtering}

The Basic Local Alignment Search Tool, BLAST \cite{altschul1990basic} is used for DNA
sequence search on a remote or local GenBank database. It is constrained to a
"search taxon", a taxonomic group in the NCBI taxonomy that is automatically
identified using the OpenTree's taxonomic Most Recent Common Ancestor (MRCA) API
\cite{mrcaAPI, rees2017automated}, as the MRCA of all ingroup taxa that is
also a named clade in the NCBI taxonomy (Figure 1).

BLAST is performed using the `blastn` algorithm \cite{camacho2009blast}
implemented in BioPython's \cite{cock2009biopython} NCBIWWW module \cite{ncbiwww}
modified to accept an alternative BLAST address.
Each sequence in the alignment is BLASTed once against all DNA sequences in GenBank.
New sequences are excluded for analysis if they 1) are not in the search taxon;
2) have an e-value above the cutoff (default to 0.00001); 3) fall outside a min
and max length threshold, defined as the proportion of the average length without
gaps of all sequences in input alignment (default values of 80% and 120%,
respectively); 4) or if they are either identical to or shorter than an existing
sequence in the input alignment and they represent the same taxon in OpenTree or
NCBI taxonomy.
An arbitrary maximum number of randomly chosen sequences per taxon are allowed
(default to 5).

Reverse, complement, and reverse-complement sequences are identified and translated
using BioPython internal functions \cite{cock2009biopython}.
Iterative cycles of BLAST searches can be performed, by blasting all new sequences
until no new ones are found. By default only one BLAST cycle is performed.

\subsection*{New DNA sequence alignment}

MUSCLE \cite{edgar2004muscle} is used to perform a profile alignment in which the
original alignment is used as a template of homology criteria to align new sequences.
The final alignment is not further automatically checked, and additional inspection
and refinement are recommended.

\subsection*{Tree reconstruction and comparison}

RAxML \cite{stamatakis2014raxml} is implemented to reconstruct a Maximum Likelihood
(ML) gene tree for each input alignment with default settings (GTRCAT model and
100 bootstrap replicates with default algorithm), using input tree as starting
tree for ML searches.
Bootstrap results are summarized using DendroPy's SumTrees module
\cite{sukumaran2010dendropy}.

Physcraper's main result is an updated phylogenetic hypothesis for the search taxon.
Updated and original tree are compared with Robinson-Foulds weighted and unweighted
metrics estimated with Dendropy \cite{sukumaran2010dendropy}, and with a node by node
comparison between the synthetic OpenTree and original and updated tree individually,
using OpenTree's conflict API \cite{redelings2017supertree}.



\section*{Results}
\subsection*{Case Study: The hollies}

A user is interested in phylogenetic relationships within the genus \textit{Ilex}. Commonly
known as "hollies", the genus encompasses between 400-700 living species, and is
the only extant clade within the family Aquifoliaceae, order Aquifoliales of flowering
plants.

An online literature review in June 2020 (Google scholar search for "ilex phylogeny")
reveals that there are several published phylogenies showing relationships within
the hollies \cite{cuenoud2000molecular, manen2010history, setoguchi2000intersectional,
selbach2009new}, but only two have data publicly available \cite{gottlieb2005molecular, yao2020phylogeny}.
\cite{gottlieb2005molecular} made original tree and alignment available in TreeBASE
(study 1091 \cite{treebase1091}). This tree (Gottlieb2005 tree from now on) samples 41 species, is available from
OpenTree's Phylesystem (study pg\_2827 \cite{pg2827}),
and has been integrated into OpenTree's synthetic tree \cite{mrcaott68451ott89474}.
The most recent \textit{Ilex} tree \cite{yao2020phylogeny} (Yao2020 tree from now on) is available in
OpenTree's Phylesystem (study ot\_1984 \cite{ot1984}),
and in the DRYAD repository \cite{yao2020dryad}.
With 175 tips, the Yao2020 tree \cite{yao2020phylogeny} is the best sampled phylogeny available for
the genus \textit{Ilex}.

We ran Physcraper on a laptop Linux computer to update an internal transcribed
spacer DNA region (ITS) alignment that was used to construct the tree from \cite{gottlieb2005molecular},
using a local GenBank database.  BLAST and RAxML analyses ran for 19hrs 45min, with bootstrap
analyses taking an additional 13hrs.
%% <!-- MTH: you probably need some details about the hardware, given the fact that you are discussing running times -->
The Gottlieb2005 tree \cite{gottlieb2005molecular} updated using Physcraper (Figure 2;
Physcraper updated tree from now on) displays all 41 distinct
taxa from the original study plus 231 new tips, contributing phylogenetic data to
84 additional \textit{Ilex} taxa. The best ML tree from the RAxML analysis is 99\%
resolved, with 25\% of nodes with bootstrap support $<$ 0.1 and 48\% nodes with
bootstrap support $>$ 0.75.
A large portion of internal branches are negligibly small, with 30 branches
$<$ 0.00001 substitution rate units, from which only 9 have a bootstrap support $>$ 0.75
(Figure 2).
As comparison with the Physraper updated tree, the Yao2020 tree \cite{yao2020phylogeny}
also contains all 41 distinct taxa
sampled in the Gottlieb2005 tree \cite{gottlieb2005molecular},
while contributing phylogenetic data to 134 additional \textit{Ilex} taxa. From these,
67 taxa are also in the Physcraper updated tree.
While the Yao2020 tree \cite{yao2020phylogeny} was also constructed using
ITS as a marker, their GenBank data is not released yet. Hence, Physcraper was unable
to incorporate 68 taxa that are only on the Yao2020 tree because the DNA data is unavailable.
We also note that Physcraper incorporates 18 \textit{Ilex} taxa that are not in the Yao2020
tree \cite{yao2020phylogeny}. These taxa appear nested among other \textit{Ilex} species
and visual inspection of the DNA sequences suggests they are correctly assigned as \textit{Ilex}.
The ITS alignment that underlies the Yao2020 tree was constructed without any tool to
scrape GenBank \cite{yao2020phylogeny}, which could explain why Physcraper was able to
incorporate these 18 additional \textit{Ilex} taxa in the updated tree (Figure 2).

\subsection*{Verification test}

We pruned 9 out of the 41 original tips of the Gottlieb et al. tree, corresponding
to a 20\% trim, excluding the outgroups. We then performed a Physcraper run to test
if we would recover the dropped tips.
We successfully recovered 6 out of 9 pruned tips in the updated tree.
Closer examination of results revealed that the 3 missing tips were correctly retrieved
with BLAST along with the other 6 tips that were included in the updated tree. Examining
the ITS sequences belonging to the tips missed by the test, by following the GenBank
accession numbers reported in the original publication, showed that these sequences
contain a 100 bp long gap of unidentified nucleotides (Ns). These Ns were absent
in the 3 sequences published in the original alignment.

\section*{Discussion}
Databases preserving and democratizing access to biological data
have become essential resources for science.
New molecular data keep accumulating and tools facilitating its integration into
existent evolutionary knowledge contribute to the acceleration of scientific discovery.

Physcraper is a tool that builds upon previous knowledge stored in published
alignments, with the main goal to extend our knowledge of phylogenetic relationships
across the tree of life.

We believe this is a key step to successfully establish an open, reproducible workflow for phylogenetics,
facilitating phylogenetic knowledge for ecologists and other non-specialists, effectively
democratizing phylogenetic studies.

As a tool for automatizing phylogenetic reconstruction from molecular databases,
Physcraper presents several advantages over existing phylogenetic pipelines designed to
make evolutionary sense of the vast amount of
public molecular data available.

Several analysis tools create full phylogenies \textit{de novo} by mining of molecular
databases \cite{antonelli2017toward, sanderson2008phylota, smith2019pyphlawd,
bennett2018phylotar, pearse2013phylogenerator}.
In particular, Phylota \cite{sanderson2008phylota},
and PHLAWD \cite{smith2009mega}, have been cited and used abundantly.

Physcraper builds on this automated database mining concept by incorporating prior
phylogenetic work and existing taxonomic domain knowledge on appropriate markers
and alignment construction.
This decreases error (requiring less manual downstream processing) and
eases comparison with previous phylogenetic
knowledge.

Results from the verification test highlight the importance of incorporating existing expertly
curated homology statements to automatically update phylogenetic relationships, instead
of ignoring the information they contain and building homology statements fully \textit{de novo}.
The three sequences that were in the original tree but were not included in
the updated tree by Physcraper in the verification test were excluded because
they contain a 100 bp long gap of unidentified nucleotides (Ns) in GenBank that
was manually curated and trimmed in the original alignment.

Unlike phylogenetic placement approaches \cite{berger_performance_2011, matsen_pplacer_2010},
which add new taxa without modifying the input tree, Physcraper estimates all
the relationships anew in the context of the new data.
PUMPER \cite{izquierdo2014pumper} shares these conceptual strengths, but is no
longer under active development, is challenging to install and run, and has resulted
in very few phylogenetic analyses since its publication.

Physcraper generates gene trees, which individually do not capture the full complexity
of species' evolutionary history \cite{song2012resolving}.
However, Physcraper facilitates
gathering alignments and gene trees for multiple loci from a group of interest,
that together can be used to reconstruct species trees taking into account coalescent
processes with ASTRAL \cite{mirarab2014astral},
BEAST2 \cite{bouckaert2019beast}, or SVD Quartets \cite{chifman2014quartet}).
Rigorous analyses of multiple loci allows for more complex evolutionary models
than analyses of large genomic data sets, and can provide better evolutionary estimates.

Physcraper has the added advantage of facilitating the linkage of taxonomic information about tips
in the output phylogenies to data available in a variety of biological databases
\cite{rees2017automated}, such as
geographical locations for taxa from the GBIF \cite{gbif_secretariat_gbif_2019}.

The Physcraper workflow can be used to rapidly (in a matter of hours) create
phylogenies which can address challenges overarching both fields of ecology and evolution, such as
phylogenetically placing newly discovered species \cite{webb2010biodiversity},
curating taxonomic assignments
\cite{san2010molecular},
and generating custom trees for ecological \cite{helmus2012phylogenetic} and
evolutionary downstream analyses \cite{stoltzfus2013phylotastic}.

\section*{Conclusions}
Data repositories hold more information than meets the eye.
Beyond the main data, they are rich sources of metadata that can be leveraged
for the advantage of all areas of biology as well as the advancement of scientific
policy and applications.
Initial ideas about the data are constantly changed by results from new analyses.
Physcraper provides a framework for reproducible phylogenetics that has the
potential to consistently provide context for these ideas, highlighting the
importance of data sharing and open science in the field, biology and science.

\section*{Availability and requirements}

\textbf{Project name:} Physcraper

\textbf{Project home page:} \href{https://physcraper.readthedocs.io/en/latest/index.html}{https://physcraper.readthedocs.io/en/latest/index.html}

\textbf{Operating System:} Linux, Mac, Windows

\textbf{Programming Language:} Python

\textbf{Other requirements:} Dependencies

\textbf{License:} GNU

\textbf{Any restrictions to use by non-academics:} As specified by the License

\section*{Abbreviations}

\textbf{OpenTree:} The Open Tree of Life project

\textbf{TNRS:} Taxonomic Name Resolution Service

\textbf{MRCA:} Most Recent Common Ancestor

\textbf{BLAST:} Basic Local Alignment Search Tool

\textbf{NCBI:} USA National Center for Biodiversity Information

\textbf{GBIF:} Global Biodiversity Information Facility

%%%%%%%%%%%%%%%%%%%%%%%%%%%%%%%%%%%%%%%%%%%%%%
%%                                          %%
%% Backmatter begins here                   %%
%%                                          %%
%%%%%%%%%%%%%%%%%%%%%%%%%%%%%%%%%%%%%%%%%%%%%%
\section*{Declarations}

\begin{backmatter}

\section*{Ethics approval and consent to participate}
Not applicable.

\section*{Consent for publication}
Not applicable.

\section*{Availability of data and materials}

The datasets generated and analysed during the current study are available in
the repositories "physcraper" containing the source code,
\href{https://github.com/McTavishLab/physcraper}{https://github.com/McTavishLab/physcraper};
"physcraperex" containing the examples,
\href{https://github.com/McTavishLab/physcraperex}{https://github.com/McTavishLab/physcraperex};
and, "physcraper\_ms" containing this reproducible manuscript,
\href{https://github.com/McTavishLab/physcraper\_ms}{https://github.com/McTavishLab/physcraper\_ms}.

\section*{Competing interests}
The authors declare that they have no competing interests.

\section*{Funding}
This research was supported by the grant "Sustaining the Open Tree of Life", NSF ABI
No. 1759838, and ABI No. 1759846.
Computer time was provided by the Multi-Environment Research Computer for
Exploration and Discovery (MERCED) cluster from the University of California,
Merced (UCM), supported by the NSF Grant No. ACI-1429783.

\section*{Authors' contributions}
LLSR wrote manuscript, alignment code, documentation, performed analyses and developed examples; MK wrote code for ncbidataparser module, filtering of sequences per OTU and using offline blast searches, wrote documentation and tests; EJM conceived study, wrote most of the code, documentation and tests.
All authors contributed to the manuscript and gave final approval for publication.
\ldots

\section*{Acknowledgements}
We thank the members of the OpenTree development team and the "short bar" Science
and Engineering Building 1, UCM, joint lab paper discussion group for valuable
comments on this manuscript.


%%%%%%%%%%%%%%%%%%%%%%%%%%%%%%%%%%%%%%%%%%%%%%%%%%%%%%%%%%%%%
%%                  The Bibliography                       %%
%%                                                         %%
%%  Bmc_mathpys.bst  will be used to                       %%
%%  create a .BBL file for submission.                     %%
%%  After submission of the .TEX file,                     %%
%%  you will be prompted to submit your .BBL file.         %%
%%                                                         %%
%%                                                         %%
%%  Note that the displayed Bibliography will not          %%
%%  necessarily be rendered by Latex exactly as specified  %%
%%  in the online Instructions for Authors.                %%
%%                                                         %%
%%%%%%%%%%%%%%%%%%%%%%%%%%%%%%%%%%%%%%%%%%%%%%%%%%%%%%%%%%%%%

% if your bibliography is in bibtex format, use those commands:
\bibliographystyle{bmc-mathphys} % Style BST file (bmc-mathphys, vancouver, spbasic).
\bibliography{bmc_article}      % Bibliography file (usually '*.bib' )
% for author-year bibliography (bmc-mathphys or spbasic)
% a) write to bib file (bmc-mathphys only)
% @settings{label, options="nameyear"}
% b) uncomment next line
%\nocite{label}

% or include bibliography directly:
% \begin{thebibliography}
% \bibitem{b1}
% \end{thebibliography}

%%%%%%%%%%%%%%%%%%%%%%%%%%%%%%%%%%%
%%                               %%
%% Figures                       %%
%%                               %%
%% NB: this is for captions and  %%
%% Titles. All graphics must be  %%
%% submitted separately and NOT  %%
%% included in the Tex document  %%
%%                               %%
%%%%%%%%%%%%%%%%%%%%%%%%%%%%%%%%%%%

%%
%% Do not use \listoffigures as most will included as separate files

\section*{Figures}
  \begin{figure}[h!]
  \caption{The Physcraper framework consists of four general steps. The software is fully
  described on its documentation website at \href{https://physcraper.readthedocs.io}{https://physcraper.readthedocs.io},
  along with installation instructions, function usage descriptions, examples and tutorials.
}
\end{figure}

\begin{figure}[h!]
  \caption{A) Phylogeny updated with Physcraper from original
  \cite{gottlieb2005molecular} tree in B.
  Tips in original alignment and new tips added with Physcraper are depicted in
  black and red, respectively. Physcraper obtained sequences from the GenBank
  database via local BLAST of all sequences in the original alignment that generated
  tree in B), filtered them following criteria from section "DNA sequence search and
  filtering", aligned them to original alignment using MUSCLE and performed a
  phylogenetic reconstruction using RAxML with 100 bootstraps. B-D conflict analyses
  performed with OpenTree tools.}
\end{figure}


\end{backmatter}
\end{document}
