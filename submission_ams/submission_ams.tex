\documentclass[draft]{ametsoc}
\usepackage{color}
\usepackage{hyperref}
\journal{jhm}
%  Please choose a journal abbreviation to use above from the following list:
%
%   jamc     (Journal of Applied Meteorology and Climatology)
%   jtech     (Journal of Atmospheric and Oceanic Technology)
%   jhm      (Journal of Hydrometeorology)
%   jpo     (Journal of Physical Oceanography)
%   jas      (Journal of Atmospheric Sciences)
%   jcli      (Journal of Climate)
%   mwr      (Monthly Weather Review)
%   wcas      (Weather, Climate, and Society)
%   waf       (Weather and Forecasting)
%   bams (Bulletin of the American Meteorological Society)
%   ei    (Earth Interactions)

%%%%%%%%%%%%%%%%%%%%%%%%%%%%%%%%
%Citations should be of the form ``author year''  not ``author, year''
\bibpunct{(}{)}{;}{a}{}{,}
\title{Title here}

\authors{Andrew N. Other
and Fred T. Secondauthor
\thanks{Current address: Some other place, Germany}}
\affiliation{American Meteorological Society,Boston, Massachusetts}


\extraauthor{Ping Lu
\correspondingauthor{Ping Lu,American Meteorological Society, 45 Beacon St., Boston, MA 02108.}
}
\extraaffil{Princeton University}
\email{groupleader@unknown.uu}
\extraauthor{Miao Yu
\thanks{Current address: Some other place, Canada}
}
\extraaffil{University of Waterloo}


% pandoc header
\abstract{some abstract}
\begin{document}
\maketitle
\hypertarget{introduction}{%
\section{Introduction}\label{introduction}}

Phylogenies are important.

Generating phylogenies is not easy and it is largely artisanal. Although
many efforts to automatize the process have been done, and the community
is using those more and more, automatization of phylogenetic
reconstruction is still not a widespread practice and among other
benefits, it might be key for adoption of better reproducibility
practices in the phylogenetics community. \textbf{\emph{paragraph better
to end discussion??? }}

The process of phylogenetic reconstruction implies many steps (that I
generalize to the following):

\begin{enumerate}
\def\labelenumi{\arabic{enumi}.}
\tightlist
\item
  Obtention of molecular or morphological character data -- get DNA from
  some organisms and sequence it, or get it from an online nucleotide
  data repository, such as GenBank (Benson et al. 2000; Wheeler et al.
  2000).
\item
  Assemble a hypothesis of homology -- Create a matrix of your character
  data, by aligning the sequences, in the case of molecular data. Make
  sure thay are paralogs!
\item
  Analyse this hypothesis of homology to infer phylogenetic
  relationships among the organisms you are studying -- Use different
  available programs to infer molecular evolution, trees and times of
  divergence.
\item
  Discuss the inferred relationships in the context of previous
  hypothesis, the biology and biogeography of the organisms, etc. --
  Answer the question, \emph{is this phylogenetic solution
  fair/reasonable?}
\end{enumerate}

Each of these steps require different types of specialized training: in
the field, in the lab, in front of a computer, discussions with experts
in the methods, and/or in the biological group of study. All of these
steps also require considerable amounts of time for training and
implementation.

In the past decade, various studies have developed solutions to
automatize the first and second steps, by creating pipelines that mine
already available molecular data from the GenBank repository (Benson et
al. 2000; Wheeler et al. 2000), to obtain homologous characters that can
be used for phylogenetic reconstruction. These tools have been presented
as aid for the nonspecialist to decrease some of the difficulties in the
generation of phylogenetic knowledge. However, they are not that often
used as so, suggesting that there are still difficulties for the
nonspecialist. The phylogenetic community has some reserves towards
these tools, too. Mainly because they sometimes act as a black box.
However, automatizing the assembly of the character data set is a
crucial step towards reproducibility for a task that was otherwise
primarily artisanal and hence largely non-reproducible.

Even if it is hard to obtain phylogenies, we invest copious amounts of
time and energy in generating them. Issues such as food security, global
warming, global health are crucial to solve and phylogenies might help.
There is a lot of phylogenetic knowledge already available in published
peer-reviewed studies. In this sense, the non-specialists (and also the
specialist) face a new problem: how do I choose the best phylogeny.

Public phylogenies can be updated with the ever increasing amount of
genetic data that is available on GenBank (Benson et al. 2000; Wheeler
et al. 2000).

We present a way to automatize and standardize the comparison of
phylogenetic hypotheses and to allow reproducibility of this last step
of the research process.

A key aspect of the standard phylogenetic workflow is comparison with
already existing phylogenetic hypotheses and with phylogenies that are
considered ``best'' by experts not only in phylogenetics, but also
experts on the focal group of study.

Concerns I think people have about these tools: - Errors in
identification of sequences - Little control along the process - Too
much of a black box?

Most of these phylogenies are being constructed by people learning about
the methods, so they want to know what is going on.

The pipelines are so powerful and they will give you an answer, but
there is no way to assess if it is better than previous answers, it just
assumes it is better because it used more data.

All these pipelines start tree construction from zero? Yes.

The goal of Physcraper is to build upon previous phylogenetic knowledge,
allowing a direct comparison between existing phylogenies and
phylogenies that are constructed using new genetic data retrieved from a
public nucleotide database (i.e., GenBank (Benson et al. 2000; Wheeler
et al. 2000)).

To achieve this, Physcraper uses the Open Tree of Life phylesystem and
connects it to the TreeBase database, to (1) get the original DNA data
set matrices (alignments) that produced a phylogeny that was published
and then made available in the OToL database, (2) use this DNA
alignments as a starting point to get new genetic data belonging to the
focal group of study, to (3) finally update the phylogenetic
relationships in the group.

A less automated workflow is one in which the alignments that generated
the published phylogeny are stored in other public database (such as
DRYAD) or elsewhere (the users computer), and are provided by the users.

The original tree is by default used as starting tree for the
phylogenetic searches, but it can also be set as a full topological
constraint or not used at all, depending on the goals of the user.

Physcraper implements node by node comparison of the the original and
the updated trees, using the conflict API of OToL.

\hypertarget{how-does-physcraper-work}{%
\section{How does Physcraper work?}\label{how-does-physcraper-work}}

\hypertarget{the-input-a-study-tree-and-an-alignment}{%
\subsection{The input: a study tree and an
alignment}\label{the-input-a-study-tree-and-an-alignment}}

\begin{itemize}
\tightlist
\item
  The study tree is a published phylogenetic tree stored in the OToL
  database, phylesystem (McTavish et al. 2015). The main reason for this
  is that trees in phylesystem have a set of user defined
  characteristics that are essential for automatizing the phylogeny
  update process. The most relevant of these being the definition of
  ingroup and outgroup. Outgroup and ingroup taxa in the original tree
  are identified and tagged. This allows to automatically set the root
  for the updated tree on the next steps of the pipeline. A user can
  choose from the `r rotl::tol\_about()\$num\_source\_trees' published
  trees supporting the resolved node of the synthetic tree in the OToL
  website (\textless{}\textgreater{}). If the tree you are interested in
  updating is not in there, you can upload it via OToL's curator tool
  (\textless{}https://tree.opentreeoflife.org/curator).
\item
  The alignment should be a gene alignment that was used to generate the
  tree. The original alignments are usually stored in a public
  repository such as TreeBase (Piel et al. 2009; Vos et al. 2012), DRYAD
  (\url{http://datadryad.org/}), or the journal were the tree was
  originally published. If the alignment is stored in TreeBase,
  \texttt{physcraper} can download it directly, either from the TreeBASE
  website (\url{https://treebase.org/}) or through the TreeBASE GitHub
  repository (SuperTreeBASE;
  \url{https://github.com/TreeBASE/supertreebase}). If the alignment is
  on another repository, or provided personally by the owner, a copy of
  it has to be downloaded by the user, and it's local path has to be
  provided as an argument.
\item
  A taxon name matching step is performed to verify that all taxon names
  on the tips of the tree are in the DNA character matrix and vice
  versa.
\item
  A ``.csv'' file with the summary of taxon name matching is produced
  for the user.
\item
  Unmatched taxon names are dropped from both the tree and alignment.
  Technically, just one matching name is needed to perform the searches.
  Please, see next section.
\item
  A ``.tre'' file and a ``.fas'' file containing only the matched taxa
  are generated and saved in the \texttt{inputs} folder to be used in
  the following steps.
\end{itemize}

\hypertarget{dna-sequence-search-and-cleaning}{%
\subsection{DNA sequence search and
cleaning}\label{dna-sequence-search-and-cleaning}}

\begin{itemize}
\item
  The next step is to identify the search taxon within the reference
  taxonomy. The search taxon will be used to constraint the DNA sequence
  search on the nucleotide database within that taxonomic group. Because
  we are using the NCBI nucleotide database, by default the reference
  taxonomy is the NCBI taxonomy. The search taxon can be provided by the
  user. If none is provided, then the search taxon is identified as the
  Most Recent Common Ancestor (MRCA) of the matched taxa belonging to
  the ingroup in the tree, that is also a named clade in the reference
  taxonomy. This is known as the Most Recent Common Ancestral Taxon
  (MRCAT; also referred in the literature as the Least Inclusive Common
  Ancestral Taxon - LICA). The MRCAT can be different from the
  phylogenetic MRCA when the latter is an unnamed clade in the reference
  taxonomy. To automatically identify the MRCAT of a group of taxon
  names, we make use of the OToL taxonomy tool
  (\url{https://github.com/OpenTreeOfLife/germinator/wiki/Taxonomy-API-v3\#mrca}).

  Users can provide a search taxon that is either a more or a less
  inclusive clade relative to the ingroup of the original phylogeny. If
  the search taxon is more inclusive, the sequence search will be
  performed outside the MRCAT of the matched taxa, e.g., including all
  taxa within the family or the order that the ingroup belongs to. If
  the search taxon is a less inclusive clade, the users can focus on
  enriching a particular clade/region within the ingroup of the
  phylogeny.
\item
  The Basic Local Alignment Search Tool, BLAST {[}Altschul et al.
  (1990); altschul1997gapped{]} is used to identify similarity between
  DNA sequences within the search taxon in a nucleotide database, and
  the remaining sequences on the alignment. The BLAST command line tools
  (Camacho et al. 2009) are used for both web-based and local-database
  searches. 
\item
  A pairwise alignment-against-all BLAST search is performed. This means
  that each sequence in the alignment is BLASTed against DNA sequences
  in a nucleotide database constrained to the search taxon. Results from
  each one of these BLAST runs are recorded, and matched sequences are
  saved along with their corresponding identification numbers (accesion
  numbers in the case of the GenBank database). This information will be
  used later to store the whole sequences in a dedicated library within
  the physcraper folder, allowing for secondary analyses to run
  significantly faster.
\item
  The DNA sequence similarity search can be done on a local database
  that is easily setup by the user. In this case, the BLASTn algorithm
  is used to performs the similarity search.
\item
  The search can also be performed remotely, on the NCBI database. In
  this case, the bioPython BLAST algorithm is used to perform the
  similarity search.
\item
  Matched sequences below an e-value, percentage similarity, and outside
  a minimum and maximum length threshold are discarded. This filtering
  leaves out genomic sequences. All acepted sequences are asigned an
  internal identifier, and are further filtered.
\item
  Because the original alignments usually lack database id numbers, a
  filtering step is needed. Accepted sequences that belong to the same
  taxon of the query sequence, and that are either identical or shorter
  than the original sequence are discarded. Only longer sequences
  belonging to the same taxon as the orignal sequence will be considered
  further for analysis.
\item
  Among the remaining filtered sequences, there are usually several
  exemplars per taxon. Although it can be useful to keep some of them
  to, for example, investigate monophyly within species, there can be
  hundreds of exemplar sequences per taxon for some markers. To control
  the number of sequences per taxon in downstream analyses, 5 sequences
  per taxon are chosen at random. This number is set by default but can
  be modified by the user.
\item
  Reverse complement sequences are identified and translated.
\item
  Users can choose to perform a more ``cycles'' of sequence similarity
  search, by blasting the newly found sequences. This can be done
  iteratively, but by default only sequences in the alignment are
  blasted. \textbf{\emph{Is there an argument to control the number of
  cycles of blast searches with new sequences?}}
\item
  Accepted sequences are downloaded in full, and stored as a local
  database in a directory that is globally accesible
  (physcraper/taxonomy), so they are accesible for further runs.
\item
  A fasta file containing all filtered and processed sequences resulting
  from the BLAST search is generated for the user.
\end{itemize}

\hypertarget{dna-sequence-alignment}{%
\subsection{DNA sequence alignment}\label{dna-sequence-alignment}}

\begin{itemize}
\tightlist
\item
  The software MUSCLE (Edgar 2004) is implemented to perform alignments.
\item
  First, all new sequences are aligned using default MUSCLE options.
\item
  Then, a MUSCLE profile alignment is performed, in which the original
  alignment is used as a template to align new sequences. This ensures
  that the final alignment follows the homology criteria established by
  the original alignment.
\item
  The final alignment is not further processed automatically. We
  encourage users to check it either by eye and perform manual
  refinement or using any of the many tools for alignment processing, to
  eliminate columns with no information.
\end{itemize}

\hypertarget{tree-reconstruction-and-comparison}{%
\subsection{Tree reconstruction and
comparison}\label{tree-reconstruction-and-comparison}}

\begin{itemize}
\tightlist
\item
  A gene tree is reconstructed for each alignment provided, using the
  software RAxML (Stamatakis 2014) with 100 classic bootstrap
  (Felsenstein 1985) replicates by default. The number of bootsrap
  replicates can be modified by the user. Other type of bootstrap that I
  think is not yet incorporated into physcraper is the Transfer
  Bootstrap Expectation (TBE) recently proposed in Lemoine et al.
  (2018).
\item
  The final result is an updated phylogenetic hypothesis for each of the
  genes provided in the alignment.
\item
  Tips on all trees generated by physcraper are defined by a taxon name
  space, allowing to perform comparisons and conflict analyses.
\item
  Robinson Foulds metrics
\item
  Describe what a conflict analysis is: Node by node comparison of the
  resulting clades compared to
\item
  For the conflict analysis to be meaningful, the root of the tree
  ineeds to be accurately defined.
\item
  Currently, the root is determined by finding the parent node of the
  sequences that do not belong to the ingroup/ search taxon. This
  ensures a correct rooting of the tree even when the search taxon is
  more inclusive than the ingroup.
\item
  Conflict information can only be generated in the context of the whole
  Open Tree of Life. Otherwise, it is not really possible to get
  conflict data. \textbf{\emph{- One way to compare two independent
  phylogenetic trees is to compare them both to the synthetic OToL and
  then measure how well they do against each other}}
\end{itemize}

\hypertarget{examples}{%
\section{Examples}\label{examples}}

\hypertarget{the-hollies}{%
\subsection{The hollies}\label{the-hollies}}

The genus \emph{Ilex} is the only extant clade within the family
Aquifoliaceae, order Aquifoliales of flowering plants. It encompasses
between 400-600 living species. A review of litterature shows that there
are three published phylogenetic trees, showing relationships within the
hollies. The first one has been made available both on OToL phylesystem
and synth tree, and on treeBASE, it samples 48 species. The second has
not been made available anywhere, not even in supplementary data of the
journal. \textbf{\emph{Contact authors? They seem old school, probably
do not wanna share their data.}} The most recent one has been made
available in the OToL Phylesystem and DRYAD. It is the best sampled yet,
with 200 species. However, it has not been added to the syntehtic tree
yet. This makes it a perfect case to test the basic functionalities of
physcraper: we know that the sequences of the most recently published
tree have been made available on the GenBank database (Benson et al.
2000; Wheeler et al. 2000). Updating the oldest tree, we should get
something very similar to the newest tree.

\hypertarget{the-ascomycota}{%
\subsection{The Ascomycota}\label{the-ascomycota}}

Let's be more specific now about our X group and say it is the
Ascomycota. The best tree currently available in OToL was published by
Schoch et al. (2009). The first step, is to get the Open Tree of Life
study id. There are some options to do this: - You can go to the Open
Tree of Life website and browse until you find it, or - you can get the
study id using R tools: - By using the TreeBase ID of the study (which
is not fully exposed on the TreeBase website home page of the study, so
you have to really look it up manually):

\begin{Shaded}
\begin{Highlighting}[]
\NormalTok{rotl}\OperatorTok{::}\KeywordTok{studies_find_studies}\NormalTok{(}\DataTypeTok{property =} \StringTok{"treebaseId"}\NormalTok{, }\DataTypeTok{value =} \StringTok{"S2137"}\NormalTok{)}
\end{Highlighting}
\end{Shaded}

\begin{verbatim}
##   study_ids n_trees tree_ids candidate study_year title
## 1    pg_238       1  tree109                 2009      
##                                 study_doi
## 1 http://dx.doi.org/10.1093/sysbio/syp020
\end{verbatim}

\begin{itemize}
\tightlist
\item
  By using the name of the focal clade of study (but this behaved very
  differently):
\end{itemize}

\begin{Shaded}
\begin{Highlighting}[]
\NormalTok{rotl}\OperatorTok{::}\KeywordTok{studies_find_studies}\NormalTok{(}\DataTypeTok{property=}\StringTok{"ot:focalCladeOTTTaxonName"}\NormalTok{, }\DataTypeTok{value=}\StringTok{"Ascomycota"}\NormalTok{)}
\end{Highlighting}
\end{Shaded}

Once we have the study id, we can gather the trees published on that
study:

\begin{Shaded}
\begin{Highlighting}[]
\NormalTok{rotl}\OperatorTok{::}\KeywordTok{get_tree_ids}\NormalTok{(rotl}\OperatorTok{::}\KeywordTok{get_study_meta}\NormalTok{(}\StringTok{"pg_238"}\NormalTok{))}
\end{Highlighting}
\end{Shaded}

\begin{verbatim}
## [1] "tree109"
\end{verbatim}

\begin{Shaded}
\begin{Highlighting}[]
\NormalTok{rotl}\OperatorTok{::}\KeywordTok{candidate_for_synth}\NormalTok{(rotl}\OperatorTok{::}\KeywordTok{get_study_meta}\NormalTok{(}\StringTok{"pg_238"}\NormalTok{))}
\end{Highlighting}
\end{Shaded}

\begin{verbatim}
## NULL
\end{verbatim}

\begin{Shaded}
\begin{Highlighting}[]
\NormalTok{my_trees <-}\StringTok{ }\NormalTok{rotl}\OperatorTok{::}\KeywordTok{get_study}\NormalTok{(}\StringTok{"pg_238"}\NormalTok{)}
\end{Highlighting}
\end{Shaded}

Both trees from this study have NA tips.

Let's check what one of the trees looks like:

\begin{enumerate}
\def\labelenumi{\arabic{enumi}.}
\tightlist
\item
  Download the alignment from TreeBase If you are on the TreeBase home
  page of the study, you can navigate to the matrix tab, and manually
  download the alignments that were used to reconstruct the trees
  reported on the study that were also uploaded to TreeBase and to the
  Open Tree of Life repository. To make this task easier, you can use a
  command to download everything into your working folder:
\end{enumerate}

\begin{verbatim}
physcraper_run.py -s pg_238 -t tree109 -o ../physcraper_example/pg_238
\end{verbatim}

In this example, all alignments posted on TreeBase were used to
reconstruct both trees.

\begin{enumerate}
\def\labelenumi{\arabic{enumi}.}
\tightlist
\item
  With the study id and the alignment files saved locally, we can do a
  physcraper run with the command:
\end{enumerate}

\begin{verbatim}
physcraper_run.py -s pg_238 -t tree109 -a treebase_alns/pg_238tree109.aln -as "nexus" -o pg_238
\end{verbatim}

\hypertarget{testudines-example}{%
\subsection{Testudines example}\label{testudines-example}}

Phylogeny of the Testudines 6 tips from Crawford et al. (2012) There is
just one tree in OToL. There is just one alignment on
\href{https://treebase.org/treebase-web/search/study/matrices.html?id=12742}{treebase}
with all the 1 145 loci.

\begin{verbatim}
physcraper_run.py -s pg_2573 -t tree5959 -tb -db ~/branchinecta/local_blast_db/ -o pg_2573
\end{verbatim}

\hypertarget{discussion}{%
\section{Discussion}\label{discussion}}

Data repositories hold more information than meets the eye. Besides the
actual data, they have other types of information that can be used for
the advantage of science.

Usually, initial ideas about the data are changed by analyses. We expect
that this new ideas on the data can be registered on data bases,
exposing new comers to expert understanding about the data.

There are many tools that are making use of DNA data repositories in
different ways. Most of them focus on efficient ways to mine the data --
getting the most homologs. Some focus on accurate ways of mining the
data - getting real and clean homologs. Others focus on refinement of
the alignment. Most focus on generating full trees \emph{de novo},
mainly for regions of the Tree of Life that have no phylogenetic
assessment yet in published studies, but also for regions that have been
already studied and that have phylogenetic data already.

All these tools are great efforts for advancing towards reproducibility
in phylogenetics, a field that has been largely recognised as somewhat
artisanal. We propose adding focus to other sources of information
available from data repositories. Taking advantage of public DNA data
bases have been the main focus. However, phylogenetic knowledge is also
accumulating fast in public and open repositories. In this way, the
physcraper pipeline can be complemented with other tools that have been
developed for other purposes.

We emphasize that physcraper takes advantage of the knowledge and
intuition of the expert community to build upon phylogenetic knowledge,
using not only data accumulated in DNA repositories, but phylogenetic
knowledge accumulated in tree repositories. This might help generate new
phylogenetic data. But physcraper does not seek to generate full
phylogenies \emph{de novo}.

Describe again statistics to compare phylogenies provided by physcraper
via OpenTreeOfLife. Mention statistics provided by other tools:
PhyloExplorer (Ranwez et al. 2009). Compare and discuss.

How is physcraper already useful: - to mine targeted sequences, in this
way it is similar to baited analyses from PHLAWD and pyPHLAWD. Phylota
does not do baited analyses, I think, only clustered analyses. - Finding

How can it be used for the advantage of the field: - rapid phylogenetic
placing of newly discovered species, as mentioned in Webb et al. (2010)
- obtain trees for ecophylogenetic studies, as mentioned in Helmus and
Ives (2012) - one day could be used to sistematize nucleotide databases,
such as Genbank (Benson et al. 2000; Wheeler et al. 2000), as mentioned
in San Mauro and Agorreta (2010), i.e., curate ncbi taxonomic
assignations. - allows to generate custom species trees for downstream
analyses, as mentioned in Stoltzfus et al. (2013)

Things that physcraper does not do: - analyse the whole GenBank database
(Benson et al. 2000; Wheeler et al. 2000) to find homolog regions
suitable to reconstruct phylogenies, as mentioned in Antonelli et al.
(2017). There are already some very good tools that do that. - provide
basic statistics on data availability to assemble molecular datasets, as
mentioned by Ranwez et al. (2009). Phyloexplorer does this? - it is not
a tree repo, as phylota is, mentioned in Deepak et al. (2014)

\hypertarget{tools-that-automatize-any-part-of-the-process-of-phylogenetic-reconstruction}{%
\subsection{Tools that automatize any part of the process of
phylogenetic
reconstruction:}\label{tools-that-automatize-any-part-of-the-process-of-phylogenetic-reconstruction}}

\hypertarget{mining-dna-databases-to-generate-datasets-suitable-for-phylogenetic-reconstruction}{%
\subsubsection{1. Mining DNA databases to generate datasets suitable for
phylogenetic
reconstruction}\label{mining-dna-databases-to-generate-datasets-suitable-for-phylogenetic-reconstruction}}

\begin{longtable}[]{@{}llccc@{}}
\toprule
\begin{minipage}[b]{0.12\columnwidth}\raggedright
Tool\strut
\end{minipage} & \begin{minipage}[b]{0.15\columnwidth}\raggedright
Citation\strut
\end{minipage} & \begin{minipage}[b]{0.20\columnwidth}\centering
Cited by\strut
\end{minipage} & \begin{minipage}[b]{0.20\columnwidth}\centering
Description\strut
\end{minipage} & \begin{minipage}[b]{0.20\columnwidth}\centering
Supermatrix/gene tree/species tree\strut
\end{minipage}\tabularnewline
\midrule
\endhead
\begin{minipage}[t]{0.12\columnwidth}\raggedright
Phylota\strut
\end{minipage} & \begin{minipage}[t]{0.15\columnwidth}\raggedright
Sanderson et al. (2008)\strut
\end{minipage} & \begin{minipage}[t]{0.20\columnwidth}\centering
122 studies\strut
\end{minipage} & \begin{minipage}[t]{0.20\columnwidth}\centering
finds sets of DNA homologs on the GenBank database; phylogenetic
reconstruction\strut
\end{minipage} & \begin{minipage}[t]{0.20\columnwidth}\centering
Supermatrix\strut
\end{minipage}\tabularnewline
\begin{minipage}[t]{0.12\columnwidth}\raggedright
AMPHORA\strut
\end{minipage} & \begin{minipage}[t]{0.15\columnwidth}\raggedright
Wu and Eisen (2008)\strut
\end{minipage} & \begin{minipage}[t]{0.20\columnwidth}\centering
458 studies\strut
\end{minipage} & \begin{minipage}[t]{0.20\columnwidth}\centering
baited search; protein markers on phylogenomic data; personal database
of genomes or metagenomic data, manually downloaded either from a public
database or from private data; phylogenetic reconstruction\strut
\end{minipage} & \begin{minipage}[t]{0.20\columnwidth}\centering
Supermatrix\strut
\end{minipage}\tabularnewline
\begin{minipage}[t]{0.12\columnwidth}\raggedright
PHLAWD\strut
\end{minipage} & \begin{minipage}[t]{0.15\columnwidth}\raggedright
Smith et al. (2009)\strut
\end{minipage} & \begin{minipage}[t]{0.20\columnwidth}\centering
234 studies\strut
\end{minipage} & \begin{minipage}[t]{0.20\columnwidth}\centering
Baited search of DNA markers on the GenBank database; phylogenetic
reconstruction\strut
\end{minipage} & \begin{minipage}[t]{0.20\columnwidth}\centering
Supermatrix\strut
\end{minipage}\tabularnewline
\begin{minipage}[t]{0.12\columnwidth}\raggedright
Unnamed
\href{https://www.zfmk.de/en/research/research-centres-and-groups/taming-of-an-impossible-child-pipeline-tools-and-manuals}{ruby
pipeline}, only available from
\href{https://static-content.springer.com/esm/art\%3A10.1186\%2F1741-7007-9-55/MediaObjects/12915_2011_480_MOESM1_ESM.ZIP}{supplementary
data} of the journal\strut
\end{minipage} & \begin{minipage}[t]{0.15\columnwidth}\raggedright
Peters et al. (2011)\strut
\end{minipage} & \begin{minipage}[t]{0.20\columnwidth}\centering
64 studies\strut
\end{minipage} & \begin{minipage}[t]{0.20\columnwidth}\centering
mining public DNA databases, focuses on filtering massive amounts of
mined sequences by using established ``criteria of compositional
homogeneity and defined levels of density and overlap''\strut
\end{minipage} & \begin{minipage}[t]{0.20\columnwidth}\centering
Supermatrix\strut
\end{minipage}\tabularnewline
\begin{minipage}[t]{0.12\columnwidth}\raggedright
Unnamed\strut
\end{minipage} & \begin{minipage}[t]{0.15\columnwidth}\raggedright
Grant and Katz (2014)\strut
\end{minipage} & \begin{minipage}[t]{0.20\columnwidth}\centering
38 studies\strut
\end{minipage} & \begin{minipage}[t]{0.20\columnwidth}\centering
predecessor of phylotol; homolog clustering; public and/or personal DNA
database; phylogenetic reconstruction; broad taxon analyses; remove
contaminant sequences, based on similarity and on phylogenetic
position\strut
\end{minipage} & \begin{minipage}[t]{0.20\columnwidth}\centering
supermatrix\strut
\end{minipage}\tabularnewline
\begin{minipage}[t]{0.12\columnwidth}\raggedright
Unnamed\strut
\end{minipage} & \begin{minipage}[t]{0.15\columnwidth}\raggedright
Chesters and Zhu (2014)\strut
\end{minipage} & \begin{minipage}[t]{0.20\columnwidth}\centering
10 studies\strut
\end{minipage} & \begin{minipage}[t]{0.20\columnwidth}\centering
algorithm that mines GenBank data to delineate species in the insecta.
The authors present a nice comparison with the phylota algorithm\strut
\end{minipage} & \begin{minipage}[t]{0.20\columnwidth}\centering
Species trees??\strut
\end{minipage}\tabularnewline
\begin{minipage}[t]{0.12\columnwidth}\raggedright
PUmPER\strut
\end{minipage} & \begin{minipage}[t]{0.15\columnwidth}\raggedright
Izquierdo-Carrasco et al. (2014)\strut
\end{minipage} & \begin{minipage}[t]{0.20\columnwidth}\centering
14 studies\strut
\end{minipage} & \begin{minipage}[t]{0.20\columnwidth}\centering
perpetual updating with newly added sequences to GenBank\strut
\end{minipage} & \begin{minipage}[t]{0.20\columnwidth}\centering
not sure yet\strut
\end{minipage}\tabularnewline
\begin{minipage}[t]{0.12\columnwidth}\raggedright
DarwinTree\strut
\end{minipage} & \begin{minipage}[t]{0.15\columnwidth}\raggedright
Meng et al. (2015)\strut
\end{minipage} & \begin{minipage}[t]{0.20\columnwidth}\centering
6 studies\strut
\end{minipage} & \begin{minipage}[t]{0.20\columnwidth}\centering
predecessor is Phylogenetic Analysis of Land Plants Platform (PALPP),
takes data from GenBank, EMBL and DDBJ for land plants only\strut
\end{minipage} & \begin{minipage}[t]{0.20\columnwidth}\centering
not sure\strut
\end{minipage}\tabularnewline
\begin{minipage}[t]{0.12\columnwidth}\raggedright
NCBIminer\strut
\end{minipage} & \begin{minipage}[t]{0.15\columnwidth}\raggedright
Xu et al. (2015)\strut
\end{minipage} & \begin{minipage}[t]{0.20\columnwidth}\centering
4 studies\strut
\end{minipage} & \begin{minipage}[t]{0.20\columnwidth}\centering
part of darwintree\strut
\end{minipage} & \begin{minipage}[t]{0.20\columnwidth}\centering
not sure\strut
\end{minipage}\tabularnewline
\begin{minipage}[t]{0.12\columnwidth}\raggedright
SUMAC\strut
\end{minipage} & \begin{minipage}[t]{0.15\columnwidth}\raggedright
Freyman (2015)\strut
\end{minipage} & \begin{minipage}[t]{0.20\columnwidth}\centering
19 studies\strut
\end{minipage} & \begin{minipage}[t]{0.20\columnwidth}\centering
both ``baited'' analyses and single‐linkage clustering methods, as well
as a novel means of determining when there are enough overlapping data
in the DNA matrix\strut
\end{minipage} & \begin{minipage}[t]{0.20\columnwidth}\centering
not sure\strut
\end{minipage}\tabularnewline
\begin{minipage}[t]{0.12\columnwidth}\raggedright
STBase\strut
\end{minipage} & \begin{minipage}[t]{0.15\columnwidth}\raggedright
McMahon et al. (2015)\strut
\end{minipage} & \begin{minipage}[t]{0.20\columnwidth}\centering
7 studies\strut
\end{minipage} & \begin{minipage}[t]{0.20\columnwidth}\centering
pipeline for species tree construction and the public database of one
million precomputed species trees\strut
\end{minipage} & \begin{minipage}[t]{0.20\columnwidth}\centering
species trees\strut
\end{minipage}\tabularnewline
\begin{minipage}[t]{0.12\columnwidth}\raggedright
Unnamed\strut
\end{minipage} & \begin{minipage}[t]{0.15\columnwidth}\raggedright
Papadopoulou et al. (2015)\strut
\end{minipage} & \begin{minipage}[t]{0.20\columnwidth}\centering
17 studies\strut
\end{minipage} & \begin{minipage}[t]{0.20\columnwidth}\centering
Automated DNA-based plant identification for large-scale biodiversity
assessment\strut
\end{minipage} & \begin{minipage}[t]{0.20\columnwidth}\centering
not sure\strut
\end{minipage}\tabularnewline
\begin{minipage}[t]{0.12\columnwidth}\raggedright
BIR\strut
\end{minipage} & \begin{minipage}[t]{0.15\columnwidth}\raggedright
Kumar et al. (2015)\strut
\end{minipage} & \begin{minipage}[t]{0.20\columnwidth}\centering
6 studies\strut
\end{minipage} & \begin{minipage}[t]{0.20\columnwidth}\centering
blast, align, identify homologs via constructed trees, curate and
realign\strut
\end{minipage} & \begin{minipage}[t]{0.20\columnwidth}\centering
supermatrix\strut
\end{minipage}\tabularnewline
\begin{minipage}[t]{0.12\columnwidth}\raggedright
SUPERSMART\strut
\end{minipage} & \begin{minipage}[t]{0.15\columnwidth}\raggedright
Antonelli et al. (2017)\strut
\end{minipage} & \begin{minipage}[t]{0.20\columnwidth}\centering
35 studies\strut
\end{minipage} & \begin{minipage}[t]{0.20\columnwidth}\centering
baited analyses up to bayesian divergence time estimation\strut
\end{minipage} & \begin{minipage}[t]{0.20\columnwidth}\centering
supermatrix\strut
\end{minipage}\tabularnewline
\begin{minipage}[t]{0.12\columnwidth}\raggedright
SOPHI\strut
\end{minipage} & \begin{minipage}[t]{0.15\columnwidth}\raggedright
{[}Chesters (2017)\strut
\end{minipage} & \begin{minipage}[t]{0.20\columnwidth}\centering
17 studies\strut
\end{minipage} & \begin{minipage}[t]{0.20\columnwidth}\centering
Searches DNA sequence data from repos other than GenBank, such as
transcriptomic and barcoding repos\strut
\end{minipage} & \begin{minipage}[t]{0.20\columnwidth}\centering
not sure\strut
\end{minipage}\tabularnewline
\begin{minipage}[t]{0.12\columnwidth}\raggedright
phyloSkeleton\strut
\end{minipage} & \begin{minipage}[t]{0.15\columnwidth}\raggedright
Guy (2017)\strut
\end{minipage} & \begin{minipage}[t]{0.20\columnwidth}\centering
5 studies\strut
\end{minipage} & \begin{minipage}[t]{0.20\columnwidth}\centering
focuses on taxon sampling; baited genomic sequences; public database
(NCBI and JGI); marker identification\strut
\end{minipage} & \begin{minipage}[t]{0.20\columnwidth}\centering
supermatrix\strut
\end{minipage}\tabularnewline
\begin{minipage}[t]{0.12\columnwidth}\raggedright
OneTwoTree\strut
\end{minipage} & \begin{minipage}[t]{0.15\columnwidth}\raggedright
Drori et al. (2018)\strut
\end{minipage} & \begin{minipage}[t]{0.20\columnwidth}\centering
7 studies\strut
\end{minipage} & \begin{minipage}[t]{0.20\columnwidth}\centering
Web‐based, user-friendly, online tool for species-tree reconstruction,
based on the \emph{supermatrix paradigm} and retrieves all available
sequence data from NCBI GenBank\strut
\end{minipage} & \begin{minipage}[t]{0.20\columnwidth}\centering
supermatrix\strut
\end{minipage}\tabularnewline
\begin{minipage}[t]{0.12\columnwidth}\raggedright
pyPhlawd\strut
\end{minipage} & \begin{minipage}[t]{0.15\columnwidth}\raggedright
Smith and Walker (2019)\strut
\end{minipage} & \begin{minipage}[t]{0.20\columnwidth}\centering
6 studies\strut
\end{minipage} & \begin{minipage}[t]{0.20\columnwidth}\centering
baited and clustering analyses\strut
\end{minipage} & \begin{minipage}[t]{0.20\columnwidth}\centering
Supermatrix or gene tree\strut
\end{minipage}\tabularnewline
\begin{minipage}[t]{0.12\columnwidth}\raggedright
Phylotol\strut
\end{minipage} & \begin{minipage}[t]{0.15\columnwidth}\raggedright
Cerón-Romero et al. (2019)\strut
\end{minipage} & \begin{minipage}[t]{0.20\columnwidth}\centering
5 studies\strut
\end{minipage} & \begin{minipage}[t]{0.20\columnwidth}\centering
``phylogenomic pipeline to allow easy incorporation of data from
high-throughput sequencing studies, to automate production of both
multiple sequence alignments and gene trees, and to identify and remove
contaminants. PhyloToL is designed for phylogenomic analyses of diverse
lineages across the tree of life'', i.e., bacteria and unicellular
eukaryotes\strut
\end{minipage} & \begin{minipage}[t]{0.20\columnwidth}\centering
supermatrix and gene trees\strut
\end{minipage}\tabularnewline
\begin{minipage}[t]{0.12\columnwidth}\raggedright
phylotaR\strut
\end{minipage} & \begin{minipage}[t]{0.15\columnwidth}\raggedright
Bennett et al. (2018)\strut
\end{minipage} & \begin{minipage}[t]{0.20\columnwidth}\centering
studies\strut
\end{minipage} & \begin{minipage}[t]{0.20\columnwidth}\centering
\strut
\end{minipage} & \begin{minipage}[t]{0.20\columnwidth}\centering
\strut
\end{minipage}\tabularnewline
\bottomrule
\end{longtable}

According to Cerón-Romero et al. (2019), PhyLoTA and BIR ``focus on the
identification and collection of homologous and paralog genes from
public databases such as GenBank'', while both AMPHORA and PHLAWD
``focus on the construction and refinement of robust alignments rather
than the collection of homologs.''

\hypertarget{searching-phylogenetic-tree-databases}{%
\subsubsection{2. Searching phylogenetic tree
databases}\label{searching-phylogenetic-tree-databases}}

PhyloFinder (Chen et al. 2008) - cited by 18: a search engine for
phylogenetic databases, using trees from TreeBASE - more related to
phylotastic's goal than to updating/creating phylogenies

\hypertarget{mining-phylogenetic-tree-databases}{%
\subsubsection{3. Mining phylogenetic tree
databases}\label{mining-phylogenetic-tree-databases}}

PhyloExplorer (Ranwez et al. 2009) - cited by 21: a python and MySQL
based website to facilitate assessment and management of phylogenetic
tree collections. It provides ``statistics describing the collection,
correcting invalid taxon names, extracting taxonomically relevant parts
of the collection using a dedicated query language, and identifying
related trees in the TreeBASE database''.

\hypertarget{pipeline-for-phylogenetic-reconstruction}{%
\subsubsection{4. Pipeline for phylogenetic
reconstruction}\label{pipeline-for-phylogenetic-reconstruction}}

PhySpeTre (Fang et al. 2019) - no citations yet - no sequence retrieval,
just phylogenetic reconstruction pipeline.

\hypertarget{getting-metadata-and-not-sequences-from-genbank.}{%
\subsubsection{5. getting metadata and not sequences from
GenBank.}\label{getting-metadata-and-not-sequences-from-genbank.}}

Datataxa Ruiz-Sanchez et al. (2019) - no citations yet - focus on
extracting metadata from GenBank sequence information.

\hypertarget{phylota-overview}{%
\subsection{Phylota overview}\label{phylota-overview}}

\hypertarget{acknowledgements}{%
\section{Acknowledgements}\label{acknowledgements}}

We acknowledge contributions from

The University of California, Merced cluster, MERCED (Multi-Environment
Research Computer for Exploration and Discovery) supported by the
National Science Foundation (Grant No.~ACI-1429783).

\hypertarget{references}{%
\section*{References}\label{references}}
\addcontentsline{toc}{section}{References}

\hypertarget{refs}{}
\leavevmode\hypertarget{ref-altschul1990basic}{}%
Altschul, S. F., W. Gish, W. Miller, E. W. Myers, and D. J. Lipman,
1990: Basic local alignment search tool. \emph{Journal of molecular
biology}, \textbf{215}, 403--410,
doi:\href{https://doi.org/10.1016/S0022-2836(05)80360-2}{10.1016/S0022-2836(05)80360-2}.

\leavevmode\hypertarget{ref-antonelli2017toward}{}%
Antonelli, A. and Coauthors, 2017: Toward a self-updating platform for
estimating rates of speciation and migration, ages, and relationships of
taxa. \emph{Systematic Biology}, \textbf{66}, 152--166,
doi:\href{https://doi.org/10.1093/sysbio/syw066}{10.1093/sysbio/syw066}.

\leavevmode\hypertarget{ref-bennett2018phylotar}{}%
Bennett, D. J., H. Hettling, D. Silvestro, A. Zizka, C. D. Bacon, S.
Faurby, R. A. Vos, and A. Antonelli, 2018: PhylotaR: An automated
pipeline for retrieving orthologous dna sequences from genbank in r.
\emph{Life}, \textbf{8}, 20,
doi:\href{https://doi.org/10.3390/life8020020}{10.3390/life8020020}.

\leavevmode\hypertarget{ref-benson2000genbank}{}%
Benson, D. A., I. Karsch-Mizrachi, D. J. Lipman, J. Ostell, B. A. Rapp,
and D. L. Wheeler, 2000: GenBank. \emph{Nucleic acids research},
\textbf{28}, 15--18,
doi:\href{https://doi.org/10.1093/nar/28.1.15}{10.1093/nar/28.1.15}.

\leavevmode\hypertarget{ref-camacho2009blast}{}%
Camacho, C., C. George, A. Vahram, M. Ning, P. Jason, B. Kevin, and L.
Thomas, 2009: BLAST+: Architecture and applications. \emph{BMC
bioinformatics}, \textbf{10}, 421,
doi:\href{https://doi.org/10.1186/1471-2105-10-421}{10.1186/1471-2105-10-421}.

\leavevmode\hypertarget{ref-ceron2019phylotol}{}%
Cerón-Romero, M. A., X. X. Maurer-Alcalá, J.-D. Grattepanche, Y. Yan, M.
M. Fonseca, and L. Katz, 2019: PhyloToL: A taxon/gene-rich phylogenomic
pipeline to explore genome evolution of diverse eukaryotes.
\emph{Molecular biology and evolution}, \textbf{36}, 1831--1842,
doi:\href{https://doi.org/10.1093/molbev/msz103}{10.1093/molbev/msz103}.

\leavevmode\hypertarget{ref-chen2008phylofinder}{}%
Chen, D., J. G. Burleigh, M. S. Bansal, and D. Fernández-Baca, 2008:
PhyloFinder: An intelligent search engine for phylogenetic tree
databases. \emph{BMC Evolutionary Biology}, \textbf{8}, 90,
doi:\href{https://doi.org/10.1186/1471-2148-8-90.}{10.1186/1471-2148-8-90.}

\leavevmode\hypertarget{ref-chesters2017construction}{}%
Chesters, D., 2017: Construction of a species-level tree of life for the
insects and utility in taxonomic profiling. \emph{Systematic biology},
\textbf{66}, 426--439,
doi:\href{https://doi.org/10.1093/sysbio/syw099}{10.1093/sysbio/syw099}.

\leavevmode\hypertarget{ref-chesters2014protocol}{}%
------, and C.-D. Zhu, 2014: A protocol for species delineation of
public dna databases, applied to the insecta. \emph{Systematic biology},
\textbf{63}, 712--725,
doi:\href{https://doi.org/10.1093/sysbio/syu038}{10.1093/sysbio/syu038}.

\leavevmode\hypertarget{ref-crawford2012more}{}%
Crawford, N. G., B. C. Faircloth, J. E. McCormack, R. T. Brumfield, K.
Winker, and T. C. Glenn, 2012: More than 1000 ultraconserved elements
provide evidence that turtles are the sister group of archosaurs.
\emph{Biology letters}, \textbf{8}, 783--786,
doi:\href{https://doi.org/10.1098/rsbl.2012.0331}{10.1098/rsbl.2012.0331}.

\leavevmode\hypertarget{ref-deepak2014evominer}{}%
Deepak, A., D. Fernández-Baca, S. Tirthapura, M. J. Sanderson, and M. M.
McMahon, 2014: EvoMiner: Frequent subtree mining in phylogenetic
databases. \emph{Knowledge and Information Systems}, \textbf{41},
559--590,
doi:\href{https://doi.org/10.1007/s10115-013-0676-0}{10.1007/s10115-013-0676-0}.

\leavevmode\hypertarget{ref-drori2018onetwotree}{}%
Drori, M., A. Rice, M. Einhorn, O. Chay, L. Glick, and I. Mayrose, 2018:
OneTwoTree: An online tool for phylogeny reconstruction. \emph{Molecular
ecology resources}, \textbf{18}, 1492--1499,
doi:\href{https://doi.org/10.1111/1755-0998.12927}{10.1111/1755-0998.12927}.

\leavevmode\hypertarget{ref-edgar2004muscle}{}%
Edgar, R. C., 2004: MUSCLE: Multiple sequence alignment with high
accuracy and high throughput. \emph{Nucleic acids research},
\textbf{32}, 1792--1797,
doi:\href{https://doi.org/10.1093/nar/gkh340}{10.1093/nar/gkh340}.

\leavevmode\hypertarget{ref-fang2019physpetree}{}%
Fang, Y., C. Liu, J. Lin, X. Li, K. N. Alavian, Y. Yang, and Y. Niu,
2019: PhySpeTree: An automated pipeline for reconstructing phylogenetic
species trees. \emph{BMC evolutionary biology}, \textbf{19}, 1--8,
doi:\href{https://doi.org/10.1186/s12862-019-1541-x}{10.1186/s12862-019-1541-x}.

\leavevmode\hypertarget{ref-felsenstein1985confidence}{}%
Felsenstein, J., 1985: Confidence intervals on phylogenetics: An
approach using bootstrap. \emph{Evolution}, \textbf{39}, 783--791.

\leavevmode\hypertarget{ref-freyman2015sumac}{}%
Freyman, W. A., 2015: SUMAC: Constructing phylogenetic supermatrices and
assessing partially decisive taxon coverage. \emph{Evolutionary
Bioinformatics}, \textbf{11}, EBO--S35384,
doi:\href{https://doi.org/10.4137/EBO.S35384}{10.4137/EBO.S35384}.

\leavevmode\hypertarget{ref-grant2014building}{}%
Grant, J. R., and L. A. Katz, 2014: Building a phylogenomic pipeline for
the eukaryotic tree of life-addressing deep phylogenies with
genome-scale data. \emph{PLoS currents}, \textbf{6},
doi:\href{https://doi.org/10.1371/currents.tol.c24b6054aebf3602748ac042ccc8f2e9}{10.1371/currents.tol.c24b6054aebf3602748ac042ccc8f2e9}.

\leavevmode\hypertarget{ref-guy2017phyloskeleton}{}%
Guy, L., 2017: PhyloSkeleton: Taxon selection, data retrieval and marker
identification for phylogenomics. \emph{Bioinformatics}, \textbf{33},
1230--1232,
doi:\href{https://doi.org/10.1093/bioinformatics/btw824}{10.1093/bioinformatics/btw824}.

\leavevmode\hypertarget{ref-helmus2012phylogenetic}{}%
Helmus, M. R., and A. R. Ives, 2012: Phylogenetic diversity--area
curves. \emph{Ecology}, \textbf{93}, S31--S43,
doi:\href{https://doi.org/10.1890/11-0435.1}{10.1890/11-0435.1}.

\leavevmode\hypertarget{ref-izquierdo2014pumper}{}%
Izquierdo-Carrasco, F., J. Cazes, S. A. Smith, and A. Stamatakis, 2014:
PUmPER: Phylogenies updated perpetually. \emph{Bioinformatics},
\textbf{30}, 1476--1477,
doi:\href{https://doi.org/10.1093/bioinformatics/btu053}{10.1093/bioinformatics/btu053}.

\leavevmode\hypertarget{ref-kumar2015bir}{}%
Kumar, S., A. K. Krabberød, R. S. Neumann, K. Michalickova, S. Zhao, X.
Zhang, and K. Shalchian-Tabrizi, 2015: BIR pipeline for preparation of
phylogenomic data. \emph{Evolutionary Bioinformatics}, \textbf{11},
EBO--S10189,
doi:\href{https://doi.org/10.4137/EBO.S10189}{10.4137/EBO.S10189}.

\leavevmode\hypertarget{ref-lemoine2018renewing}{}%
Lemoine, F., J.-B. D. Entfellner, E. Wilkinson, D. Correia, M. D.
Felipe, T. De Oliveira, and O. Gascuel, 2018: Renewing felsenstein's
phylogenetic bootstrap in the era of big data. \emph{Nature},
\textbf{556}, 452--456.

\leavevmode\hypertarget{ref-mcmahon2015stbase}{}%
McMahon, M. M., A. Deepak, D. Fernández-Baca, D. Boss, and M. J.
Sanderson, 2015: STBase: One million species trees for comparative
biology. \emph{PloS one}, \textbf{10},
doi:\href{https://doi.org/10.1371/journal.pone.0117987}{10.1371/journal.pone.0117987}.

\leavevmode\hypertarget{ref-mctavish2015phylesystem}{}%
McTavish, E. J., C. E. Hinchliff, J. F. Allman, J. W. Brown, K. A.
Cranston, M. T. Holder, J. A. Rees, and S. A. Smith, 2015: Phylesystem:
A git-based data store for community-curated phylogenetic estimates.
\emph{Bioinformatics}, \textbf{31}, 2794--2800,
doi:\href{https://doi.org/10.1093/bioinformatics/btv276}{10.1093/bioinformatics/btv276}.

\leavevmode\hypertarget{ref-meng2015darwintree}{}%
Meng, Z., H. Dong, J. Li, Z. Chen, Y. Zhou, X. Wang, and S. Zhang, 2015:
Darwintree: A molecular data analysis and application environment for
phylogenetic study. \emph{Data Science Journal}, \textbf{14},
doi:\href{https://doi.org/10.5334/dsj-2015-010}{10.5334/dsj-2015-010}.

\leavevmode\hypertarget{ref-papadopoulou2015automated}{}%
Papadopoulou, A. and Coauthors, 2015: Automated dna-based plant
identification for large-scale biodiversity assessment. \emph{Molecular
ecology resources}, \textbf{15}, 136--152,
doi:\href{https://doi.org/10.1111/1755-0998.12256}{10.1111/1755-0998.12256}.

\leavevmode\hypertarget{ref-peters2011taming}{}%
Peters, R. S., B. Meyer, L. Krogmann, J. Borner, K. Meusemann, K.
Schütte, O. Niehuis, and B. Misof, 2011: The taming of an impossible
child: A standardized all-in approach to the phylogeny of hymenoptera
using public database sequences. \emph{BMC biology}, \textbf{9}, 55,
doi:\href{https://doi.org/10.1186/1741-7007-9-55}{10.1186/1741-7007-9-55}.
\url{https://bmcbiol.biomedcentral.com/articles/10.1186/1741-7007-9-55\#Sec21}.

\leavevmode\hypertarget{ref-piel2009treebase}{}%
Piel, W., L. Chan, M. Dominus, J. Ruan, R. Vos, and V. Tannen, 2009:
Treebase v. 2: A database of phylogenetic knowledge. E-biosphere.

\leavevmode\hypertarget{ref-ranwez2009phyloexplorer}{}%
Ranwez, V., N. Clairon, F. Delsuc, S. Pourali, N. Auberval, S. Diser,
and V. Berry, 2009: PhyloExplorer: A web server to validate, explore and
query phylogenetic trees. \emph{BMC evolutionary biology}, \textbf{9},
108,
doi:\href{https://doi.org/10.1186/1471-2148-9-108}{10.1186/1471-2148-9-108}.

\leavevmode\hypertarget{ref-ruiz2019datataxa}{}%
Ruiz-Sanchez, E., C. A. Maya-Lastra, V. W. Steinmann, S. Zamudio, E.
Carranza, R. M. Murillo, and J. Rzedowski, 2019: Datataxa: A new script
to extract metadata sequence information from genbank, the flora of
bajı́o as a case study. \emph{Botanical Sciences}, \textbf{97}, 754--760,
doi:\href{https://doi.org/10.17129/botsci.2226}{10.17129/botsci.2226}.

\leavevmode\hypertarget{ref-sanderson2008phylota}{}%
Sanderson, M. J., D. Boss, D. Chen, K. A. Cranston, and A. Wehe, 2008:
The PhyLoTA Browser: Processing GenBank for Molecular Phylogenetics
Research. \emph{Systematic Biology}, \textbf{57}, 335--346,
doi:\href{https://doi.org/10.1080/10635150802158688}{10.1080/10635150802158688}.
\url{https://doi.org/10.1080/10635150802158688}.

\leavevmode\hypertarget{ref-san2010molecular}{}%
San Mauro, D., and A. Agorreta, 2010: Molecular systematics: A synthesis
of the common methods and the state of knowledge. \emph{Cellular \&
Molecular Biology Letters}, \textbf{15}, 311,
doi:\href{https://doi.org/10.2478/s11658-010-0010-8}{10.2478/s11658-010-0010-8}.

\leavevmode\hypertarget{ref-schoch2009ascomycota}{}%
Schoch, C. L. and Coauthors, 2009: The ascomycota tree of life: A
phylum-wide phylogeny clarifies the origin and evolution of fundamental
reproductive and ecological traits. \emph{Systematic biology},
\textbf{58}, 224--239.

\leavevmode\hypertarget{ref-smith2019pyphlawd}{}%
Smith, S. A., and J. F. Walker, 2019: PyPHLAWD: A python tool for
phylogenetic dataset construction. \emph{Methods in Ecology and
Evolution}, \textbf{10}, 104--108.

\leavevmode\hypertarget{ref-smith2009mega}{}%
------, J. M. Beaulieu, and M. J. Donoghue, 2009: Mega-phylogeny
approach for comparative biology: An alternative to supertree and
supermatrix approaches. \emph{BMC evolutionary biology}, \textbf{9}, 37.

\leavevmode\hypertarget{ref-stamatakis2014raxml}{}%
Stamatakis, A., 2014: RAxML version 8: A tool for phylogenetic analysis
and post-analysis of large phylogenies. \emph{Bioinformatics},
\textbf{30}, 1312--1313,
doi:\href{https://doi.org/10.1093/bioinformatics/btu033}{10.1093/bioinformatics/btu033}.

\leavevmode\hypertarget{ref-stoltzfus2013phylotastic}{}%
Stoltzfus, A. and Coauthors, 2013: Phylotastic! Making tree-of-life
knowledge accessible, reusable and convenient. \emph{BMC
bioinformatics}, \textbf{14}, 158,
doi:\href{https://doi.org/10.1186/1471-2105-14-158}{10.1186/1471-2105-14-158}.

\leavevmode\hypertarget{ref-vos2012nexml}{}%
Vos, R. A. and Coauthors, 2012: NeXML: Rich, extensible, and verifiable
representation of comparative data and metadata. \emph{Systematic
biology}, \textbf{61}, 675--689,
doi:\href{https://doi.org/10.1093/sysbio/sys025}{10.1093/sysbio/sys025}.

\leavevmode\hypertarget{ref-webb2010biodiversity}{}%
Webb, C. O., J. F. Slik, and T. Triono, 2010: Biodiversity inventory and
informatics in southeast asia. \emph{Biodiversity and Conservation},
\textbf{19}, 955--972,
doi:\href{https://doi.org/10.1007/s10531-010-9817-x}{10.1007/s10531-010-9817-x}.

\leavevmode\hypertarget{ref-wheeler2000database}{}%
Wheeler, D. L., C. Chappey, A. E. Lash, D. D. Leipe, T. L. Madden, G. D.
Schuler, T. A. Tatusova, and B. A. Rapp, 2000: Database resources of the
national center for biotechnology information. \emph{Nucleic acids
research}, \textbf{28}, 10--14,
doi:\href{https://doi.org/doi.org/10.1093/nar/28.1.10}{doi.org/10.1093/nar/28.1.10}.

\leavevmode\hypertarget{ref-wu2008simple}{}%
Wu, M., and J. A. Eisen, 2008: A simple, fast, and accurate method of
phylogenomic inference. \emph{Genome biology}, \textbf{9}, R151,
doi:\href{https://doi.org/10.1186/gb-2008-9-10-r151}{10.1186/gb-2008-9-10-r151}.

\leavevmode\hypertarget{ref-xu2015ncbiminer}{}%
Xu, X., D. Dimitrov, C. Rahbek, and Z. Wang, 2015: NCBIminer: Sequences
harvest from genbank. \emph{Ecography}, \textbf{38}, 426--430,
doi:\href{https://doi.org/10.1111/ecog.01055}{10.1111/ecog.01055}.
\end{document}
