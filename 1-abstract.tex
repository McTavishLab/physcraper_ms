\PassOptionsToPackage{unicode=true}{hyperref} % options for packages loaded elsewhere
\PassOptionsToPackage{hyphens}{url}
%
\documentclass[]{article}
\usepackage{lmodern}
\usepackage{amssymb,amsmath}
\usepackage{ifxetex,ifluatex}
\usepackage{fixltx2e} % provides \textsubscript
\ifnum 0\ifxetex 1\fi\ifluatex 1\fi=0 % if pdftex
  \usepackage[T1]{fontenc}
  \usepackage[utf8]{inputenc}
  \usepackage{textcomp} % provides euro and other symbols
\else % if luatex or xelatex
  \usepackage{unicode-math}
  \defaultfontfeatures{Ligatures=TeX,Scale=MatchLowercase}
\fi
% use upquote if available, for straight quotes in verbatim environments
\IfFileExists{upquote.sty}{\usepackage{upquote}}{}
% use microtype if available
\IfFileExists{microtype.sty}{%
\usepackage[]{microtype}
\UseMicrotypeSet[protrusion]{basicmath} % disable protrusion for tt fonts
}{}
\IfFileExists{parskip.sty}{%
\usepackage{parskip}
}{% else
\setlength{\parindent}{0pt}
\setlength{\parskip}{6pt plus 2pt minus 1pt}
}
\usepackage{hyperref}
\hypersetup{
            pdftitle={Physcraper: A Python package for continually updated gene trees},
            pdfauthor={Luna L. Sanchez Reyes; Martha Kandziora; Emily Jane McTavish; University of California, Merced},
            pdfborder={0 0 0},
            breaklinks=true}
\urlstyle{same}  % don't use monospace font for urls
\usepackage[margin=1in]{geometry}
\usepackage{longtable,booktabs}
% Fix footnotes in tables (requires footnote package)
\IfFileExists{footnote.sty}{\usepackage{footnote}\makesavenoteenv{longtable}}{}
\usepackage{graphicx,grffile}
\makeatletter
\def\maxwidth{\ifdim\Gin@nat@width>\linewidth\linewidth\else\Gin@nat@width\fi}
\def\maxheight{\ifdim\Gin@nat@height>\textheight\textheight\else\Gin@nat@height\fi}
\makeatother
% Scale images if necessary, so that they will not overflow the page
% margins by default, and it is still possible to overwrite the defaults
% using explicit options in \includegraphics[width, height, ...]{}
\setkeys{Gin}{width=\maxwidth,height=\maxheight,keepaspectratio}
\setlength{\emergencystretch}{3em}  % prevent overfull lines
\providecommand{\tightlist}{%
  \setlength{\itemsep}{0pt}\setlength{\parskip}{0pt}}
\setcounter{secnumdepth}{5}
% Redefines (sub)paragraphs to behave more like sections
\ifx\paragraph\undefined\else
\let\oldparagraph\paragraph
\renewcommand{\paragraph}[1]{\oldparagraph{#1}\mbox{}}
\fi
\ifx\subparagraph\undefined\else
\let\oldsubparagraph\subparagraph
\renewcommand{\subparagraph}[1]{\oldsubparagraph{#1}\mbox{}}
\fi

% set default figure placement to htbp
\makeatletter
\def\fps@figure{htbp}
\makeatother


\title{Physcraper: A Python package for continually updated gene trees}
\author{Luna L. Sanchez Reyes \and Martha Kandziora \and Emily Jane McTavish \and University of California, Merced}
\date{24 June, 2020}

\begin{document}
\maketitle

{
\setcounter{tocdepth}{2}
\tableofcontents
}
\hypertarget{abstract}{%
\section{Abstract}\label{abstract}}

\begin{enumerate}
\def\labelenumi{\arabic{enumi}.}
\item
  Phylogenies are a key part of research in all areas of biology. Tools that automatize
  some parts of the process of phylogenetic reconstruction (mainly character matrix construction)
  have been developed for the advantage of both specialists in the field of phylogenetics and nonspecialists.
  However, interpretation of results, comparison with previously available phylogenetic
  hypotheses, and choosing of one phylogeny for downstream analyses and discussion still impose difficulties
  to one that is not a specialist either on phylogenetic methods or on a particular group of study.
\item
  Physcraper is an open‐source, command-line Python program that automatizes the update of published
  phylogenies by making use of public DNA sequence data and taxonomic information,
  providing a framework for comparison of published phylogenies with their updated versions.
\item
  Physcraper can be used by the nonspecialist, as a tool to generate phylogenetic
  hypothesis based on already available expert phylogenetic knowledge.
  Phylogeneticists and group specialists will find it useful as a tool to facilitate comparison
  of alternative phylogenetic hypotheses (topologies).
  \textbf{\emph{Is physcraper intended for the nonspecialist?? We have two types of nonspecialists:
  the ones that do not know about phylogenetic methods and the ones that might know
  about phylogenetic methods but do not know much about a certain biological group.}}
\item
  Physcraper implements node by node/topology comparison of the the original and the updated
  trees using the conflict API of OToL, and summarizes differences.
\item
  We hope the physcraper workflow demonstrates the benefits of opening results in phylogenetics and encourages researchers to strive for better data sharing practices.
\item
  Physcraper can be used with any OS. Detailed instructions for installation and
  use are available at \url{https://github.com/McTavishLab/physcraper}.
\end{enumerate}

\end{document}
